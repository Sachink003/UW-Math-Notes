\documentclass{report}
\usepackage[utf8]{inputenc}
\usepackage{amsmath,amssymb,amsthm, enumerate,nicefrac,fancyhdr,hyperref,graphicx,adjustbox}
\usepackage{graphicx}

%% commands
%% useful macros [add to them as needed]
% sets
\newcommand{\C}{{\mathbb{C}}} 
\newcommand{\N}{{\mathbb{N}}}
\newcommand{\Q}{{\mathbb{Q}}}
\newcommand{\R}{{\mathbb{R}}}
\newcommand{\Z}{{\mathbb{Z}}}
\newcommand{\F}{{\mathbb{F}}}

% bases
\newcommand{\mA}{\mathcal{A}}
\newcommand{\mB}{\mathcal{B}}
\newcommand{\mC}{\mathcal{C}}
\newcommand{\mD}{\mathcal{D}}
\newcommand{\mE}{\mathcal{E}}
\newcommand{\mL}{\mathcal{L}}
\newcommand{\mM}{\mathcal{M}}
\newcommand{\mO}{\mathcal{O}}
\newcommand{\mP}{\mathcal{P}}
\newcommand{\mS}{\mathcal{S}}
\newcommand{\mT}{\mathcal{T}}

% linear algebra
\newcommand{\diag}{\operatorname{diag}}
\newcommand{\adj}{\operatorname{adj}}
\newcommand{\rank}{\operatorname{rank}}
\newcommand{\spn}{\operatorname{Span}}
\newcommand{\proj}{\operatorname{proj}}
\newcommand{\prp}{\operatorname{perp}}
\newcommand{\refl}{\operatorname{refl}}
\newcommand{\tr}{\operatorname{tr}}
\newcommand{\nul}{\operatorname{Null}}
\newcommand{\nully}{\operatorname{nullity}}
\newcommand{\range}{\operatorname{Range}}
\renewcommand{\ker}{\operatorname{Ker}}
\newcommand{\col}{\operatorname{Col}}
\newcommand{\row}{\operatorname{Row}}
\newcommand{\cof}{\operatorname{cof}}
\newcommand{\Num}{\operatorname{Num}}
\newcommand{\Id}{\operatorname{Id}}
\newcommand{\ipb}{\langle \thinspace, \rangle}
\newcommand{\ip}[2]{\left\langle #1, #2\right\rangle} % inner products
\newcommand{\M}[2]{M_{#1\times #2}(\F)}
\newcommand{\RREF}{\operatorname{RREF}}
\newcommand{\cv}[1]{\begin{bmatrix} #1 \end{bmatrix}}

% vectors
\newcommand{\vzero}{\vv{0}}
\newcommand{\va}{\vv{a}}
\newcommand{\vb}{\vv{b}}
\newcommand{\vc}{\vv{c}}
\newcommand{\vd}{\vv{d}}
\newcommand{\ve}{\vv{e}}
\newcommand{\vf}{\vv{f}}
\newcommand{\vg}{\vv{g}}
\newcommand{\vh}{\vv{h}}
\newcommand{\vl}{\vv{\ell}}
\newcommand{\vm}{\vv{m}}
\newcommand{\vn}{\vv{n}}
\newcommand{\vp}{\vv{p}}
\newcommand{\vq}{\vv{q}}
\newcommand{\vr}{\vv{r}}
\newcommand{\vs}{\vv{s}}
\newcommand{\vt}{\vv{t}}
\newcommand{\vu}{\vv{u}}
\newcommand{\vvv}{{\vv{v}}}
\newcommand{\vw}{\vv{w}}
\newcommand{\vx}{\vv{x}}
\newcommand{\vy}{\vv{y}}
\newcommand{\vz}{\vv{z}}

% display
\newcommand{\ds}{\displaystyle}
\newcommand{\qand}{\quad\text{and}}
\newcommand{\qandq}{\quad\text{and}\quad}
\newcommand{\hint}{\textbf{Hint: }}

% misc
\newcommand{\area}{\operatorname{area}}
\newcommand{\vol}{\operatorname{vol}}
\newcommand{\rc}{{\color{red} \checkmark}}



\begin{document}

\title{MATH 136: Linear Algebra 1\\
{Example Sheets}}
\author{M. Sachin Kumar}
\date{Fall 2022}

\maketitle

\chapter*{UW Example Sheet 1}
\begin{enumerate}
    \item \begin{enumerate}
        \item Given that $\begin{bmatrix} a \\ 2 \\ 3 \end{bmatrix} + 2 \begin{bmatrix} 4 \\ 5 \\ 6 \end{bmatrix} = \frac{1}{2} \begin{bmatrix} 8 \\ b \\ 12 \end{bmatrix} - \begin{bmatrix} 3 \\ 4 \\ c \end{bmatrix}$, determine the values of a, b, and c.
        \item Let $\vec{v} = \begin{bmatrix} 2 + i\\2 \end{bmatrix}$ and $\vec{w} = \begin{bmatrix} 3 + 6i \\ -3 - 3i\end{bmatrix}$. Determine $\vec{u} \in \C^2$ such that $9\vec{v} - 3i\vec{u} = -i\vec{w}$.
    \end{enumerate}
    \item Given two unit vectors in $\R^3$ that are orthogonal to both $\vec{v} = \begin{bmatrix} 1 \\ 3 \\ 5\end{bmatrix}$ and $\vec{w} = \begin{bmatrix} -1 \\ 1 \\ 3\end{bmatrix}$. Prove that these are the only two unit vectors in $\R^3$ that are orthogonal to both $\vec{v}$ and $\vec{w}$.
    \item Let $n \in \N$. Prove or disprove the following statements.
    \begin{enumerate}
        \item $\forall \vec{v}, \vec{w} \in \R^n$ with $\vec{v} \ne \vec{0}$ and $\vec{w} \ne \vec{0}$, $\proj_{\vec{w}}(\text{perp}_{\vec{v}}(\vec{w})) = \vec{0}$.
        \item $\forall \vec{v}, \vec{w} \in \R^n$ with $\vec{w} \ne \vec{0}$, $\proj_{\vec{w}}(\text{perp}_{\vec{w}}(\vec{v})) = \vec{0}$.
    \end{enumerate}
    \item Let $\vec{v}, \vec{w} \in \R^3$. Prove that $||\vec{v} \times \vec{w}||^2 = ||\vec{v}||^2 ||\vec{w}||^2 - (\vec{v} \cdot \vec{w})^2$.
    \item \begin{enumerate}
        \item Disprove the following statement,
        $$ \forall c \in \C \text{ and all } \vec{u}, \vec{v} \in \C^2, \langle \vec{u}, c\vec{v}\rangle = c \langle \vec{u}, \vec{v} \rangle $$
        \item Prove the Conjugate Linearity property,
        $$ \forall c \in \C \text{ and all } \vec{u}, \vec{v}, \vec{w} \in \C^n, \langle \vec{u}, \vec{v} + c \vec{w} \rangle = \langle \vec{u}, \vec{v} \rangle + c \langle \vec{u}, \vec{w} \rangle $$
    \end{enumerate}
\end{enumerate}

\chapter*{UW Example Sheet 2}
\begin{enumerate}
    \item Let $(-1, 2)$ and $(2, -3)$ be two points on a line $\mathcal{L}$ in $\R^2$.
    \begin{enumerate}
        \item Find a vector equation for $\mathcal{L}$
        \item find a parametric equation for $\mathcal{L}$
        \item Determine whether the point $(3, -4)$ is on $\mathcal{L}$.
    \end{enumerate}
    \item Prove that $\text{span} \left \{\begin{bmatrix} 1 \\ 1 \\ 1\end{bmatrix}, \begin{bmatrix} 1 \\ 0\\ 1\end{bmatrix}\right \} = \text{span} \left \{\begin{bmatrix} 2 \\ 1 \\ 2\end{bmatrix}, \begin{bmatrix} 1 \\ 0\\ 1\end{bmatrix}\right \} \in \R^3$\\
    (\textbf{Hint}: to prove two sets are equal, you must show that they are subsets of each other.)
    \item Let $\mathcal{L}_1$ and $\mathcal{L}_2$ be two line in $\R^3$ with respect to vector equations: 
    $$\vec{\ell}_1 = \begin{bmatrix} 1 \\ 2\\ 5\end{bmatrix} + t\begin{bmatrix} 2 \\ 4\\ -3\end{bmatrix} \text{ and } \vec{\ell}_2 = \begin{bmatrix} 1 \\ 2\\ 4\end{bmatrix} + t\begin{bmatrix} 2 \\ -3\\ 4\end{bmatrix}, t \in \R$$
    Prove that $\mathcal{L}_1$ and $\mathcal{L}_2$ do not intersect.
    \item The points $A = (3, -2, 1)$ and $B = (-2, 4, 3)$ lie on a plane $\mathcal{P}$.
    \begin{enumerate}
        \item Prove that the point $C = (-7, 10, 5)$ must also lie on $\mathcal{P}$.
        \item Given that the line with vector equation $\vec{\ell} = t \begin{bmatrix} \frac{5}{2}\\ -3 \\ -1\end{bmatrix}, t \in \R$ also lies on $\mathcal{P}$, find a scalar equation of $\mathcal{P}$.
    \end{enumerate}
    \item \begin{enumerate}
        \item Find $a$ and $b$ such that $\begin{bmatrix}
        2a \\ 4\end{bmatrix} \notin \text{Span} \bigg\{\begin{bmatrix}
        1 \\ 1\end{bmatrix}, \begin{bmatrix}
        3 \\ b\end{bmatrix} \bigg \} \in \R^2, \forall a, b \in \R$
        \item Determine the solution set, $S$, to the following system of linear equations.
        \begin{align*}
        x_1 + 3x_2 + 4x_4 &= 0\\
        x_1 + 3x_2 - x_3 + 3x_4 &= 0
        \end{align*}
        Express $S$ as the span of one or more vectors.
    \end{enumerate}
\end{enumerate}

\chapter*{UW Example Sheet 3}
\begin{enumerate}
    \item Consider the following systems of linear equations:
    \begin{align*}
        -3x_1 - 6x_2 - 3x_3 &= 3\\
        2x_1 + 4x_2 + 3x_3 &= -4\\
        -4x_1 - 8x_2 - 3x_3 &= 2
    \end{align*}
    \begin{enumerate}
        \item Give the coefficient matrix and the augmented matrix of this system.
        \item Determine the reduced row echelon form of the augmented matrix.
        \item State the rank of the coefficient matrix and the rank of the augmented matrix. State the nullity of the coefficient matrix.
        \item Determine the solution set for the system.
    \end{enumerate}
    \item Solve the following system of linear equations.
    \begin{align*}
        iz_1 + (1+i)z_2 &= -5 - 3i\\
        2iz_1 + (3 + 2i)z_2 + 2iz_3 &= -15\\
        3z_1 + (3 - 3i)z_2 + z_3 &= -6 + 16i\\
        3iz_1 + (4 + 3i)z_2 + 2iz_3 &= -20 - 3i
    \end{align*}
    \item A system of linear equations has the following augmented matrix.
    $$\begin{bmatrix}
    1 & -3 & 4 &\bigm| & 7\\
    0 & 2a & -4a &\bigm| & 6a\\
    0 & 0 & b^2 - 25 &\bigm| &b+5
    \end{bmatrix}$$
    $\forall a,b \in \R$. Determine all values of $a$ and $b$.
    \begin{enumerate}
        \item is inconsistent
        \item has a unique solution
        \item has infinitely many solutions.
    \end{enumerate}
    \item Let $\vec{v}_1, \dots, \vec{v}_k \in \R^n$. Assume that $\text{Span} \{\vec{v}_1, \dots, \vec{v}_k\} = \R^n$. Prove that $n \le k$, i.e., $\R^n $ cannot be spanned by fewer than $n$ vectors.
    \begin{enumerate}
        \item Let $\vec{v}_1 = \begin{bmatrix} v_{11} \\ v_{12} \\ \vdots \\ v_{1n} \end{bmatrix}, \vec{v}_2 = \begin{bmatrix} v_{21} \\ v_{22} \\ \vdots \\ v_{2n} \end{bmatrix}, \dots, \vec{v}_k = \begin{bmatrix} v_{k1} \\ v_{k2} \\ \vdots \\ v_{kn} \end{bmatrix}$, then
        \begin{align*}
            v_{11}x_1 + v_{21}x_2 + \dots + v_{k1}x_k &= b_1\\
            v_{12}x_1 + v_{22}x_2 + \dots + v_{k2}x_k &= b_2\\
            \vdots\\
            v_{1n}x_1 + v_{2n}x_2 + \dots + v_{kn}x_k &= b_n
        \end{align*}
        Prove that the system of LE's is consistent, $\forall b_1, b_2, \dots, b_n \in \R$.
        \item Using part (a), prove that $n \le k$.
    \end{enumerate}
    \item Let $\vec{v}_1, \vec{v}_2, \vec{v}_3 \in \R^2$ be distinct vectors. Prove or disprove the following statements.
    \begin{enumerate}
        \item If $\{ \vec{v}_1, \vec{v}_2, \vec{v}_3 \}$ is a spanning set for $\R^2$, then every vector in $\R^2$ can be expressed as a linear combination of $\vec{v}_1, \vec{v}_2$ and $\vec{v}_3$ in a unique way.
        \item If $\{\vec{v}_1, \vec{v}_2\}$ is a spanning set of $\R^2$, then every vector in $\R^2$ can be expressed as a linear combination of $\vec{v}_1$ and $\vec{v}_2$ in a unique way.
    \end{enumerate}
\end{enumerate}

\chapter*{UW Example Sheet 4}
\begin{enumerate}
    \item Consider $A = \begin{bmatrix}
    2 & -1 & 3 & s\\
    0 & 2 & -3 & 1\\
    -3 & 4 & 1 & 2
    \end{bmatrix}, \vec{b} = \begin{bmatrix}
    2\\t\\0\\-2
    \end{bmatrix}$ and $\vec{c} =
    \begin{bmatrix}
    -7\\0\\u
    \end{bmatrix}$
    \begin{enumerate}
        \item Given that $A\vec{b} = \vec{c}$, determine the value of $s, t$ and $u$.
        \item Use your answer from part (a) to write $\vec{c}$ as a linear combination of the columns of $A$.
    \end{enumerate}
    \item Consider the system of linear systems
    \begin{align*}
        x_1 - x_2 - x_3 + 3x_4 &= 2\\
        2x_1 - x_2 - 3x_3 + 4x_4 &= 6\\
        x_1 - 2x_3 + x_4 &= 4
    \end{align*}
    The solution set of this system of equations is, 
    $$\Bigg\{\begin{bmatrix} 4\\2\\0\\0 \end{bmatrix} + s\begin{bmatrix} 4\\2\\2\\0 \end{bmatrix} + t\begin{bmatrix} -2\\4\\0\\2 \end{bmatrix}: \forall s, t \in \F \Bigg\}$$
    \item Prove that, $\forall n \in \N$ 
    $$\begin{bmatrix} 3 & 2\\ -1 & 0 \end{bmatrix}^2 = \begin{bmatrix} 2^{n + 1} & 2^{n + 1}\\ 1 & 2 \end{bmatrix} - \begin{bmatrix} 1 & 2\\ 2^n & 2^n \end{bmatrix}$$
    \textbf{Hint}: Use proof by induction
    \item \begin{enumerate}
        \item Let $A \in M_{n \times n}(\F)$ be such that $A^2 = \mathcal{O}_{n \times n}$. Prove that $\text{Col}(A) \subseteq \text{Null}(A)$
        \item Let $Q \in M_{n \times n}(\R)$ be such that $Q^TQ = I_n$, and let $\vec{u}, \vec{v} \in \R^n$. Prove that $\vec{u}$ is orthogonal to $\vec{v}$ if and only if $Q\vec{u}$ is orthogonal to $Q\vec{v}$. \\
    \textbf{Hint}: Consider the product $\vec{u}^T\vec{v}$ of the $1 \times n$ and $n \times 1$ matrices $\vec{u}^T$ and $\vec{v}$.
    \end{enumerate}
    \item Let $A \in M_{m \times n}$ and let $\vec{b} \in \R^m$. prove that consistent system of linear equations $A\vec{x} = \vec{b}$ has a unique solution if and only if $\text{Null}(A) = \{\vec{0}\}$.
\end{enumerate}

\chapter*{UW Example Sheet 5}
\begin{enumerate}
    \item \begin{enumerate}
        \item Determine the kernel of the linear transformation $T_1: \R^3 \to \R^3$ defined by
        $$T_1\Bigg(\begin{bmatrix} x_1 \\ x_2 \\ x_3 \end{bmatrix}\Bigg) = \begin{bmatrix}
        x_1 + 2x_2 + 3x_3\\
        x_1 + 2x_2 \\
        x_1\end{bmatrix}$$
        \item Is $T_1$ one-to-one?
        \item Determine the range of the linear transformation $T_2: \R^3 \to \R^2$ defined by,
        $$T_1\Bigg(\begin{bmatrix} x_1 \\ x_2 \\ x_3 \end{bmatrix}\Bigg) = \begin{bmatrix}
        x_1 + x_2 \\
        x_3\end{bmatrix}$$
        \item Is $T_2$ onto?
    \end{enumerate}
    \item Let $T: \R^2 \to \R^2$ be the linear transformation defined by a projection onto the line with vector equation $\vec{\ell} = t \begin{bmatrix} 1 \\ 1\end{bmatrix}, t \in \R$ followed by a counter-clockwise rotation about the origin by an angle of $\dfrac{\pi}{6}$.
    \begin{enumerate}
        \item Determine the standard matrix of $T$
        \item Use the standard matrix to find the image of $\begin{bmatrix} 2 \\ 2\end{bmatrix}$ under the transformation $T$.
    \end{enumerate}
    \item Prove or Disprove the following statements
    \begin{enumerate}
        \item The transformation $T: \R^2 \to \R^2$ defined by $T \bigg(\begin{bmatrix} x \\ y\end{bmatrix}\bigg) = \begin{bmatrix} x^2 \\ y^2 \end{bmatrix}$ is linear.
        \item Let $A, B \in M_{n \times n}(\F)$. If $ A $ and $ B $ are invertible, then $ AB $ is invertible and 
        $$(AB)^{-1} = B^{-1}A^{-1}$$
    \end{enumerate}
    \item Let $\vec{a} = \begin{bmatrix} a_1 \\ \vdots \\ a_n\end{bmatrix}, \vec{b} = \begin{bmatrix} b_1 \\ \vdots \\ b_n\end{bmatrix} \in \R^n$ (two fixed vectors). Let $T: \R^n \to \R^2$ be defined as $ T(\vec{x}) = \begin{bmatrix} \vec{a} \cdot \vec{x} \\ \vec{b} \cdot \vec{x}\end{bmatrix}$
    \begin{enumerate}
        \item Prove that $ T $ is linear
        \item Find all choices of $\vec{a}$ and $\vec{b}$ such that $T$ is onto. (show proof)
        \item Prove that if $n \ge 3$, then for all choices of $\vec{a}$ and $\vec{b}$, $T$ is not one-to-one.
    \end{enumerate}
    \item Let $T: \F^n \to \F^m$ be a transformation.We say that $T$ is invertible if there exists a transformation $S: \F^m \to \F^n$ such that 
    $$\forall \vec{x} \in \F^n, (S \circ T)(\vec{x}) = \vec{x} $$
    and 
    $$\forall \vec{x} \in \F^m, (T \circ S)(\vec{x}) = \vec{x} $$
    \begin{enumerate}
        \item Prove that if $S$ exists as defined above, then $S$ is unique.
        \item Assume now that $T$ is a linear transformation. Prove that if $S$ exists as defined above, then $S$ must be a linear transformation.
    \end{enumerate}
\end{enumerate}

\end{document}