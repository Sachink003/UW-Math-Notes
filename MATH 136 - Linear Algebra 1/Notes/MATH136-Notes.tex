\documentclass[10pt]{article} 

\usepackage{fullpage}
\usepackage{bookmark}
\usepackage{amsmath}
\usepackage{amssymb}
\usepackage[dvipsnames]{xcolor}
\usepackage{hyperref} % for the URL
\usepackage[shortlabels]{enumitem}
\usepackage{mathtools}
\usepackage[most]{tcolorbox}
\usepackage[amsmath,standard,thmmarks]{ntheorem} 
\usepackage{physics}
\usepackage{pst-tree} % for the trees
\usepackage{verbatim} % for comments, for version control
\usepackage{tabu}
\usepackage{tikz}
\usepackage{float}
\usepackage{etoolbox}

\lstnewenvironment{python}{
\lstset{frame=tb,
language=Python,
aboveskip=3mm,
belowskip=3mm,
showstringspaces=false,
columns=flexible,
basicstyle={\small\ttfamily},
numbers=none,
numberstyle=\tiny\color{Green},
keywordstyle=\color{Violet},
commentstyle=\color{Gray},
stringstyle=\color{Brown},
breaklines=true,
breakatwhitespace=true,
tabsize=2}
}
{}

\lstnewenvironment{cpp}{
\lstset{
backgroundcolor=\color{white!90!NavyBlue},   % choose the background color; you must add \usepackage{color} or \usepackage{xcolor}; should come as last argument
basicstyle={\scriptsize\ttfamily},        % the size of the fonts that are used for the code
breakatwhitespace=false,         % sets if automatic breaks should only happen at whitespace
breaklines=true,                 % sets automatic line breaking
captionpos=b,                    % sets the caption-position to bottom
commentstyle=\color{Gray},    % comment style
deletekeywords={...},            % if you want to delete keywords from the given language
escapeinside={\%*}{*)},          % if you want to add LaTeX within your code
extendedchars=true,              % lets you use non-ASCII characters; for 8-bits encodings only, does not work with UTF-8
% firstnumber=1000,                % start line enumeration with line 1000
frame=single,	                   % adds a frame around the code
keepspaces=true,                 % keeps spaces in text, useful for keeping indentation of code (possibly needs columns=flexible)
keywordstyle=\color{Cyan},       % keyword style
language=c++,                 % the language of the code
morekeywords={*,...},            % if you want to add more keywords to the set
% numbers=left,                    % where to put the line-numbers; possible values are (none, left, right)
% numbersep=5pt,                   % how far the line-numbers are from the code
% numberstyle=\tiny\color{Green}, % the style that is used for the line-numbers
rulecolor=\color{black},         % if not set, the frame-color may be changed on line-breaks within not-black text (e.g. comments (green here))
showspaces=false,                % show spaces everywhere adding particular underscores; it overrides 'showstringspaces'
showstringspaces=false,          % underline spaces within strings only
showtabs=false,                  % show tabs within strings adding particular underscores
stepnumber=2,                    % the step between two line-numbers. If it's 1, each line will be numbered
stringstyle=\color{GoldenRod},     % string literal style
tabsize=2,	                   % sets default tabsize to 2 spaces
title=\lstname}                   % show the filename of files included with \lstinputlisting; also try caption instead of title
}
{}

% floor, ceiling, set
\DeclarePairedDelimiter{\ceil}{\lceil}{\rceil}
\DeclarePairedDelimiter{\floor}{\lfloor}{\rfloor}
\DeclarePairedDelimiter{\set}{\lbrace}{\rbrace}
\DeclarePairedDelimiter{\iprod}{\langle}{\rangle}

\DeclareMathOperator{\Int}{int}
\DeclareMathOperator{\mean}{mean}

% commonly used sets
\DeclareMathOperator{\N}{{\mathbb{N}}}
\DeclareMathOperator{\Q}{{\mathbb{Q}}}
\DeclareMathOperator{\Z}{{\mathbb{Z}}}
\DeclareMathOperator{\R}{{\mathbb{R}}}
\DeclareMathOperator{\C}{{\mathbb{C}}}
\DeclareMathOperator{\F}{{\mathbb{F}}}

\newcommand{\mbf}[1]{{\boldmath\bfseries #1}}

% proof implications
\newcommand{\imp}[2]{($#1\Rightarrow#2$)\hspace{0.2cm}}
\newcommand{\impe}[2]{($#1\Leftrightarrow#2$)\hspace{0.2cm}}
\newcommand{\impr}{{($\Rightarrow$)\hspace{0.2cm}}}
\newcommand{\impl}{{($\Leftarrow$)\hspace{0.2cm}}}

% align macros
\newcommand{\agspace}{\ensuremath{\phantom{--}}}
\newcommand{\agvdots}{\ensuremath{\hspace{0.16cm}\vdots}}

% convenient brackets
\newcommand{\brac}[1]{\ensuremath{\left\langle #1 \right\rangle}}

% arrows
\newcommand{\lto}[0]{\ensuremath{\longrightarrow}}
\newcommand{\fto}[1]{\ensuremath{\xrightarrow{\scriptstyle{#1}}}}
\newcommand{\hto}[0]{\ensuremath{\hookrightarrow}}
\newcommand{\mapsfrom}[0]{\mathrel{\reflectbox{\ensuremath{\mapsto}}}}

\DeclareMathOperator{\Ann}{Ann}
\DeclareMathOperator{\Aut}{Aut}
\DeclareMathOperator{\chr}{char}
\DeclareMathOperator{\coker}{coker}
\DeclareMathOperator{\disc}{disc}
\DeclareMathOperator{\End}{End}
\DeclareMathOperator{\Fix}{Fix}
\DeclareMathOperator{\Frac}{Frac}
\DeclareMathOperator{\Gal}{Gal}
\DeclareMathOperator{\GL}{GL}
\DeclareMathOperator{\Hom}{Hom}
\DeclareMathOperator{\id}{id}
\DeclareMathOperator{\im}{im}
\DeclareMathOperator{\lcm}{lcm}
\DeclareMathOperator{\Nil}{Nil}
\DeclareMathOperator{\Spec}{Spec}
\DeclareMathOperator{\spn}{span}
\DeclareMathOperator{\Stab}{Stab}
\DeclareMathOperator{\Tor}{Tor}

\newcommand{\sset}{\subseteq}

\theoremstyle{break}
\theorembodyfont{\upshape}

\newtheorem{thm}{Theorem}[subsection]
\tcolorboxenvironment{thm}{
enhanced jigsaw,
colframe=Dandelion,
colback=White!90!Dandelion,
drop fuzzy shadow east,
rightrule=2mm,
sharp corners,
before skip=10pt,after skip=10pt
}

\newtheorem{cor}{Corollary}[thm]
\tcolorboxenvironment{cor}{
boxrule=0pt,
boxsep=0pt,
colback={White!90!RoyalPurple},
enhanced jigsaw,
borderline west={2pt}{0pt}{RoyalPurple},
sharp corners,
before skip=10pt,
after skip=10pt,
breakable
}

\newtheorem{lem}[thm]{Lemma}
\tcolorboxenvironment{lem}{
enhanced jigsaw,
colframe=Red,
colback={White!95!Red},
rightrule=2mm,
sharp corners,
before skip=10pt,after skip=10pt
}

\newtheorem{ex}[thm]{Example}
\tcolorboxenvironment{ex}{% from ntheorem
blanker,left=5mm,
sharp corners,
before skip=10pt,after skip=10pt,
borderline west={2pt}{0pt}{Gray}
}

\newtheorem*{pf}{Proof}
\tcolorboxenvironment{pf}{% from ntheorem
breakable,blanker,left=5mm,
sharp corners,
before skip=10pt,after skip=10pt,
borderline west={2pt}{0pt}{NavyBlue!80!white}
}

\newtheorem{defn}{Definition}[subsection]
\tcolorboxenvironment{defn}{
enhanced jigsaw,
colframe=Cerulean,
colback=White!90!Cerulean,
drop fuzzy shadow east,
rightrule=2mm,
sharp corners,
before skip=10pt,after skip=10pt
}

\newtheorem{prop}[thm]{Proposition}
\tcolorboxenvironment{prop}{
boxrule=0pt,
boxsep=0pt,
colback={White!90!Green},
enhanced jigsaw,
borderline west={2pt}{0pt}{Green},
sharp corners,
before skip=10pt,
after skip=10pt,
breakable
}

\setlength\parindent{0pt}
\setlength{\parskip}{2pt}

\newcommand{\subject}{MATH 136 \\ Linear Algbera 1}
\newcommand{\semester}{Fall 2022}
\newcommand{\professor}{Burcu Tuncer Karabina}

\begin{document}
\let\ref\Cref

\title{\subject}
\author{Sachin Kumar\thanks{\itshape skmuthuk@uwaterloo.ca}\\ University of Waterloo}
\date{\semester\thanks{Last updated: \today}}

\maketitle
\newpage
\tableofcontents
% \listoffigures
% \listoftables
\newpage






% main document ----------------------------------------------------------

% -------------------- CHAPTER 1 -------------------------
\section{Eigenvalues and Diagonalization}

\subsection{Eigenpair}
\begin{defn}[Eigenvector, Eigenvalue and Eigenpair]
    Let $A\in M_{n \times n}(\mathbb{F})$, a non-zero vector $\vec{x}$ is an eigenvector of $A$ over $\mathbb{F}$, if there exists a scalar $\lambda \in \mathbb{F}$ such that 
    $$A\vec{x} = \lambda \vec{x}$$
    The scalar $\lambda$  is then called an eigenvalue of $A$ over $\mathbb{F}$, and the pair $(\lambda, \vec{x})$ is an eigenpair of $A$ over $\mathbb{F}$.
\end{defn}
Note:
Given any matix $A \in M_{n \times n}(\mathbb{F})$, we will seek vectors $\vec{x} $ such that $A\vec{x}$ is a scalar multiple of $\vec{x}$. For any such matrix $A, $ $\vec{x} = \vec{0}$ is such a vector because $A\vec{0} = \vec{0} = c\vec{0}$ for any constant $c$, i.e., $A \vec{0}$ is always a scalar multiple of $\vec{0}$. This fact is so trivial that we focus our attention on vectors $\vec{x} \ne \vec{0}$ such that $A \vec{x} $ is a scalar multiple of $\vec{x}$.

\subsection{Characteristic Polynomial and Eigenvalue}
\begin{defn}[Eigenvalue Equation or Eigenvalue Problem]
    Let $A\in M_{n \times n}(\mathbb{F})$, we refer to the equation $$A\vec{x} = \lambda \vec{x}$$
    or  $$(A - \lambda I)\vec{x} = \vec{0}$$ as the eigenvalue equation for the matrix $A$ over $\mathbb{F}$. It is also sometimes refered to as the eigenvalue problem. 
\end{defn}
Note: This is an unusual equation to solve since we want to solve it for both the vector $\vec{x} \in \mathbb{F}^n$ and the scalaer $\lambda \in \mathbb{F}$. We will approach the problem by first identifying eligible values of $\lambda $. We can then determine corresponding sets of vectors $\vec{x}$ that solve the equation for each $\lambda $ we identify. 
\\
As mentioned in the previous note, a trivial solution to this equation is $\vec{x} = \vec{0}$. We try to obtain a non-trivial $(\vec{x} \ne \vec{0})$ solution to the eigenvalue equation. This is possible if and only if the RREF of matrix $A - \lambda I$ has fewer than $n $ pivots, which occurs if and only if it is not invertible, i.e, $\det(A - \lambda I) = 0$.
\\ \; \\
Computing $\det(A - \lambda I) = \det \begin{bmatrix} a_{11} - \lambda & a_{12} & \dots & a_{1n}\\
    a_{21} & a_{22} - \lambda & \dots & a_{2n}\\
    \vdots & \vdots & \ddots & \vdots \\
    a_{n1} & a_{n2} & \dots & a_{nn} - \lambda \end{bmatrix}$ by expanding along the first row, we sum up $n $ terms, each of which is the product of entries in $A - \lambda I$. Since each entry is either a constant or a linear term in a variable $\lambda $, and all the products present in the expanded determinant calculation contains $n$ terms, we can infer that $\det(A - \lambda I)$ is a polynomila in $\lambda $ of degree $n$.

\begin{defn}[Characteristic Polynomial and Characteristic Equation]
    Let $A\in M_{n \times n}(\mathbb{F})$ and $\lambda \in \mathbb{F}$, the characteristic polynomial of $A$, denoted by $C_A(\lambda)$, is 
    $$C_A(\lambda) = \det(A - \lambda I)$$
    The characteristic equation of $A$ is $$C_A(\lambda) = 0$$
\end{defn}
Note: The eigenvalues of $A$ over $\mathbb{F}$ are the roots of the characteristic polynomial of $A$ in $\mathbb{F}$.
If $C_A(\lambda)$ has no real roots then $A $ will have neither eigenvalues nor eigenpairs over $\mathbb{F}$.
\subsubsection{Properties of the Characteristic Polynomial}
\begin{prop}
    Let $A \in M_{n \times n}(\mathbb{F})$. Then $A$ is invertible if and only if $\lambda = 0$ is not an eigenvalue of $A$.
\end{prop}
\begin{pf}
    \begin{align*}
        \text{A is invertible } 
        &\text{iff }\det(A) \ne 0\\
        &\text{iff }\det(A - 0I_n) \ne 0\\
        &\text{iff }0 \text{ is not a root of the characteristic polynomial}\\
        &\text{iff } 0 \text{ is not an eigenvalue of the matrix } A
    \end{align*}
\end{pf}
\begin{defn}[Trace]
  Let $A \in M_{n \times n}(\mathbb{F})$. we define the trace of $A$ by, $$\text{tr}(A) = \sum^n_{i=1}a_{ii}$$
\end{defn}
\begin{prop}[Features of the characteristic polynomial]
  Let $A \in M_{n \times n}(\mathbb{F})$ have the characteristic polynomial $C_A(\lambda) = \det(A - \lambda I)$. Then $C_A(\lambda)$ is the degree $n$ polynomial in $\lambda $ of the form
  $$C_A(\lambda) = c_n\lambda^n + c_{n-1}\lambda^{n-1} + \dots + c_1\lambda + c_0$$
  where, 
  \begin{enumerate}
    \item $c_n = (-1)^n$
    \item $c_{n-1} = (-1)^{n-1}\space \text{tr}(A)$
    \item $c_0 = \det(A)$
  \end{enumerate}
\end{prop}
\begin{pf}
  By performing cofactor expansion along the first row of $A$ and along the first row of every subsequent $(1, 1)$-submatrix, we obtain an expression of the form, 
  $$C_A(\lambda) = \det(A - \lambda I) = (a_{11} - \lambda)(a_{22} - \lambda) \dots (a_{nn} - \lambda) + f(\lambda)$$
  where $f$ is the polynomial that is the sum of the other terms of the determinant. In the above, we have singled out the contribution of the product of the diagonal entries, and have grouped all other terms. This latter grouping $f (\lambda)$ is a polynomial of degree at most $n - 2$. To see this, note that unless the cofactor calculation is done with respect to a diagonal entry, it will use an entry $(i, j)$ that corresponds to deleting the entry $(a_{ii} - \lambda)$ from the $i$-th row and the entry $(a_{jj} - \lambda)$ from the $j$-th column. This leaves us with at most $n - 2$  entries that contain a $\lambda$.
  \\ 
  Expanding out the first term in the expression $C_A(\lambda)$ to obtain,
  $$C_A{\lambda} = a_{11} \dots a_{nn} + \dots + (-1)^{n - 1}(a_{11} + \dots + a_{nn})\lambda^{n - 1} + (-1)^n\lambda ^n + f(\lambda)$$
  we see that $C_A(\lambda)$ is a polynomial of degree $n$ in $\lambda$ because we have proved that this degree is not exceeded by the polynomial $f$, which is of degree at most $n - 2$.
  \begin{enumerate}
    \item The co-efficient of $\lambda^n$ is $c_n = (-1)^n$
    \item The co-efficient of $\lambda ^{n-1}$ is $$c_{n-1} = (-1)^{n -1} (a_{11} + \dots + a_{nn}) = (-1)^{n-1}\space \text{tr}(A)$$
    \item From the definition of $C_A({\lambda})$, we have the constant term is, $$c_0 = C_A(0) = \det(A - 0 I) = \det(A)$$
  \end{enumerate}
\end{pf}
\begin{prop}[Characteristic Polynomial and Eigenvalues over $\mathbb{C}$]
  Let $A \in M_{n \times n}(\mathbb{F})$ have the characteristic polynomial, $$C_A(\lambda) = c_n\lambda^n + c_{n-1}\lambda^{(n-1)} + \dots + c_1\lambda + c_0$$
  and $n$ eigenvalues $\lambda_1, \lambda_2, \dots, \lambda_n \in \mathbb{C}$. Then $$c_{n - 1} = (-1)^{(n-1)} \sum^{n}_{i = 1}\lambda_i$$ and $$c_0 = \prod^n_{i = 1}\lambda_i$$
\end{prop}
Note that if $A$ has repeated eigenvalues over $\mathbb{C}$, then we include each eigenvalue in the list $\lambda_1, \lambda_2, \dots, \lambda_n$ as many times as its corresponding linear factor appears in the characteristic polynomial $C_A(\lambda)$.
\begin{pf}
  The eigenvalues of $A$ over $\mathbb{C}$ are the $n$ complex roots of the characteristic polynomial, so its characteristic polynomial has the form, $$C_A(\lambda) = k(\lambda - \lambda_1)(\lambda - \lambda_2) \dots (\lambda - \lambda_n)$$
  for some $k \in \C$
  \begin{enumerate}
    \item Consider $c_{n-1}$, the coefficient of $\lambda^{n-1}$ in $C_A(\lambda)$. By expanding out the expression for $C_A(\lambda)$, we find that there are $n$ terms involving $\lambda^{n-1}$, each of which is the product of $(-1)^n$ with one of the constants $-\lambda_1, -\lambda_2, \dots, -\lambda_n$ and with $\lambda$ from each of the other $n-1$ linear factors. By taking the sum of these terms, we find that 
          \begin{align*}
              c_{n-1}\lambda^n &= (-1)^n [-\lambda_1(\lambda)^{n-1} -\lambda_2(\lambda)^{n-1} - \dots - \lambda_n(\lambda)^{n-1}]\\
                      &= \Bigg((-1)^{(n-1)} \sum^{n}_{i=1} \lambda_i\Bigg)\lambda^n
          \end{align*}
          i.e., $$c_{n-1} = (-1)^{(n-1)}\sum^{n}_{i=1}\lambda_i$$
    \item Expanding the terms of $C_A{\lambda}$, its constant term must be $(-1)^n$ times the product of the constant terms in each of the $n$ linear factors, which are $-\lambda_1, -\lambda_2, \dots, -\lambda_n$. Therefore, 
          $$c_0 = (-1)^n \prod^n_{i=1}(-\lambda_i) = (-1)^{2n}\prod^n_{i=1}\lambda_i = \prod^n_{i=1}\lambda_i$$
  \end{enumerate}
\end{pf}
Excercise 1: Prove the following corollary 1 on Eigenvalues and trace/determinant\\
Let $A \in M_{n \times n}(\mathbb{F})$  have $n$ eigenvalues $\lambda_1, \lambda_2, \dots, \lambda_n \in \mathbb{C}$. Show that 
$$\sum^n_{i=1}\lambda_i = \text{tr}(A)$$ and $$\prod^n_{i=1}\lambda_i = \det(A)$$

\subsection{Eigenvectors}
Once we have found an eigenvalue $\lambda$ of a matrix $A$ over $\mathbb{F}$, we can examine the eigenvalue equation in the form $(A - \lambda I)\vec{x} = 0$  in order to obtain a corresponding eigenvector x.
\\ \; \\
Note: Consider $(\lambda, \vec{v})$ is an eigenpair of $A$. We can get other eigenpairs by pairing non-zero scalar multiples of the eigenvector $\vec{v}$ with the eigenvalue $\lambda$.

\subsection{Eigenspaces}
\begin{prop}[Linear combinations of Eigenvectors]
  Let $c, d \in \mathbb{F}$ and suppose that $(\lambda_1, \vec{x})$ and $(\lambda_1, \vec{y})$ are eigenpairs of the matrix $A$ over $\mathbb{F}$ with the same eigenvalue $\lambda_1$. if $c\vec{x} + d\vec{y} \ne \vec{0}$, then $(\lambda_1, c\vec{x} + d\vec{y})$ is also an eigenpair for $A$ with the eigenvalue $\lambda_1$.
\end{prop}
\begin{pf}
  We see that $$A(c\vec{x} + d\vec{y}) = c(A\vec{x}) + d(A \vec{y}) = c(\lambda_1\vec{x}) + d(\lambda_1\vec{y}) = \lambda_1(c\vec{x} + d\vec{y})$$ Thus, since $c\vec{x} + d\vec{y} \ne 0$, it follows that $c\vec{x} + d\vec{y}$ is an eigenvector of $A$ with eigenvalue $\lambda_1$i.e., $(\lambda_1, c\vec{x} + d\vec{y})$ is an eigenpair for $A$.
\end{pf}
Since every non-zero linear combination of eigenvectors with respect to a fixed eigenvalue $\lambda$ are also eigenvectors with respect to $\lambda$, this suggests collecting all possible eigenvectors of $A $ with respect to a value $\lambda $ into a set.
\begin{defn}[Eigenspace]
  
\end{defn}
\end{document}
