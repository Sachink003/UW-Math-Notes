\documentclass[10pt]{article} 

\usepackage{fullpage}
\usepackage{bookmark}
\usepackage{amsmath}
\usepackage{amssymb}
\usepackage[dvipsnames]{xcolor}
\usepackage{hyperref} % for the URL
\usepackage[shortlabels]{enumitem}
\usepackage{mathtools}
\usepackage[most]{tcolorbox}
\usepackage[amsmath,standard,thmmarks]{ntheorem} 
\usepackage{physics}
\usepackage{pst-tree} % for the trees
\usepackage{verbatim} % for comments, for version control
\usepackage{tabu}
\usepackage{tikz}
\usepackage{float}
\usepackage{etoolbox}

\lstnewenvironment{python}{
\lstset{frame=tb,
language=Python,
aboveskip=3mm,
belowskip=3mm,
showstringspaces=false,
columns=flexible,
basicstyle={\small\ttfamily},
numbers=none,
numberstyle=\tiny\color{Green},
keywordstyle=\color{Violet},
commentstyle=\color{Gray},
stringstyle=\color{Brown},
breaklines=true,
breakatwhitespace=true,
tabsize=2}
}
{}

\lstnewenvironment{cpp}{
\lstset{
backgroundcolor=\color{white!90!NavyBlue},   % choose the background color; you must add \usepackage{color} or \usepackage{xcolor}; should come as last argument
basicstyle={\scriptsize\ttfamily},        % the size of the fonts that are used for the code
breakatwhitespace=false,         % sets if automatic breaks should only happen at whitespace
breaklines=true,                 % sets automatic line breaking
captionpos=b,                    % sets the caption-position to bottom
commentstyle=\color{Gray},    % comment style
deletekeywords={...},            % if you want to delete keywords from the given language
escapeinside={\%*}{*)},          % if you want to add LaTeX within your code
extendedchars=true,              % lets you use non-ASCII characters; for 8-bits encodings only, does not work with UTF-8
% firstnumber=1000,                % start line enumeration with line 1000
frame=single,	                   % adds a frame around the code
keepspaces=true,                 % keeps spaces in text, useful for keeping indentation of code (possibly needs columns=flexible)
keywordstyle=\color{Cyan},       % keyword style
language=c++,                 % the language of the code
morekeywords={*,...},            % if you want to add more keywords to the set
% numbers=left,                    % where to put the line-numbers; possible values are (none, left, right)
% numbersep=5pt,                   % how far the line-numbers are from the code
% numberstyle=\tiny\color{Green}, % the style that is used for the line-numbers
rulecolor=\color{black},         % if not set, the frame-color may be changed on line-breaks within not-black text (e.g. comments (green here))
showspaces=false,                % show spaces everywhere adding particular underscores; it overrides 'showstringspaces'
showstringspaces=false,          % underline spaces within strings only
showtabs=false,                  % show tabs within strings adding particular underscores
stepnumber=2,                    % the step between two line-numbers. If it's 1, each line will be numbered
stringstyle=\color{GoldenRod},     % string literal style
tabsize=2,	                   % sets default tabsize to 2 spaces
title=\lstname}                   % show the filename of files included with \lstinputlisting; also try caption instead of title
}
{}

% floor, ceiling, set
\DeclarePairedDelimiter{\ceil}{\lceil}{\rceil}
\DeclarePairedDelimiter{\floor}{\lfloor}{\rfloor}
\DeclarePairedDelimiter{\set}{\lbrace}{\rbrace}
\DeclarePairedDelimiter{\iprod}{\langle}{\rangle}

\DeclareMathOperator{\Int}{int}
\DeclareMathOperator{\mean}{mean}

% commonly used sets
\DeclareMathOperator{\N}{{\mathbb{N}}}
\DeclareMathOperator{\Q}{{\mathbb{Q}}}
\DeclareMathOperator{\Z}{{\mathbb{Z}}}
\DeclareMathOperator{\R}{{\mathbb{R}}}
\DeclareMathOperator{\C}{{\mathbb{C}}}
\DeclareMathOperator{\F}{{\mathbb{F}}}

\newcommand{\mbf}[1]{{\boldmath\bfseries #1}}

% proof implications
\newcommand{\imp}[2]{($#1\Rightarrow#2$)\hspace{0.2cm}}
\newcommand{\impe}[2]{($#1\Leftrightarrow#2$)\hspace{0.2cm}}
\newcommand{\impr}{{($\Rightarrow$)\hspace{0.2cm}}}
\newcommand{\impl}{{($\Leftarrow$)\hspace{0.2cm}}}

% align macros
\newcommand{\agspace}{\ensuremath{\phantom{--}}}
\newcommand{\agvdots}{\ensuremath{\hspace{0.16cm}\vdots}}

% convenient brackets
\newcommand{\brac}[1]{\ensuremath{\left\langle #1 \right\rangle}}

% arrows
\newcommand{\lto}[0]{\ensuremath{\longrightarrow}}
\newcommand{\fto}[1]{\ensuremath{\xrightarrow{\scriptstyle{#1}}}}
\newcommand{\hto}[0]{\ensuremath{\hookrightarrow}}
\newcommand{\mapsfrom}[0]{\mathrel{\reflectbox{\ensuremath{\mapsto}}}}

\DeclareMathOperator{\Ann}{Ann}
\DeclareMathOperator{\Aut}{Aut}
\DeclareMathOperator{\chr}{char}
\DeclareMathOperator{\coker}{coker}
\DeclareMathOperator{\disc}{disc}
\DeclareMathOperator{\End}{End}
\DeclareMathOperator{\Fix}{Fix}
\DeclareMathOperator{\Frac}{Frac}
\DeclareMathOperator{\Gal}{Gal}
\DeclareMathOperator{\GL}{GL}
\DeclareMathOperator{\Hom}{Hom}
\DeclareMathOperator{\id}{id}
\DeclareMathOperator{\im}{im}
\DeclareMathOperator{\lcm}{lcm}
\DeclareMathOperator{\Nil}{Nil}
\DeclareMathOperator{\Spec}{Spec}
\DeclareMathOperator{\spn}{span}
\DeclareMathOperator{\Stab}{Stab}
\DeclareMathOperator{\Tor}{Tor}

\newcommand{\sset}{\subseteq}

\theoremstyle{break}
\theorembodyfont{\upshape}

\newtheorem{thm}{Theorem}[subsection]
\tcolorboxenvironment{thm}{
enhanced jigsaw,
colframe=Dandelion,
colback=White!90!Dandelion,
drop fuzzy shadow east,
rightrule=2mm,
sharp corners,
before skip=10pt,after skip=10pt
}

\newtheorem{cor}{Corollary}[thm]
\tcolorboxenvironment{cor}{
boxrule=0pt,
boxsep=0pt,
colback={White!90!RoyalPurple},
enhanced jigsaw,
borderline west={2pt}{0pt}{RoyalPurple},
sharp corners,
before skip=10pt,
after skip=10pt,
breakable
}

\newtheorem{lem}[thm]{Lemma}
\tcolorboxenvironment{lem}{
enhanced jigsaw,
colframe=Red,
colback={White!95!Red},
rightrule=2mm,
sharp corners,
before skip=10pt,after skip=10pt
}

\newtheorem{ex}[thm]{Example}
\tcolorboxenvironment{ex}{% from ntheorem
blanker,left=5mm,
sharp corners,
before skip=10pt,after skip=10pt,
borderline west={2pt}{0pt}{Gray}
}

\newtheorem*{pf}{Proof}
\tcolorboxenvironment{pf}{% from ntheorem
breakable,blanker,left=5mm,
sharp corners,
before skip=10pt,after skip=10pt,
borderline west={2pt}{0pt}{NavyBlue!80!white}
}

\newtheorem{defn}{Definition}[subsection]
\tcolorboxenvironment{defn}{
enhanced jigsaw,
colframe=Cerulean,
colback=White!90!Cerulean,
drop fuzzy shadow east,
rightrule=2mm,
sharp corners,
before skip=10pt,after skip=10pt
}

\newtheorem{prop}[thm]{Proposition}
\tcolorboxenvironment{prop}{
boxrule=0pt,
boxsep=0pt,
colback={White!90!Green},
enhanced jigsaw,
borderline west={2pt}{0pt}{Green},
sharp corners,
before skip=10pt,
after skip=10pt,
breakable
}

\setlength\parindent{0pt}
\setlength{\parskip}{2pt}

\newcommand{\subject}{MATH 138 \\ Theorems}
\newcommand{\semester}{Winter 2023}
\newcommand{\professor}{Eddie Dupont}

\begin{document}
\let\ref\Cref

\title{\subject}
\author{Sachin Kumar\thanks{\itshape skmuthuk@uwaterloo.ca}\\ University of Waterloo}
\date{\semester\thanks{Last updated: \today}}

\maketitle
\newpage
\tableofcontents
% \listoffigures
% \listoftables
\newpage


% main document ----------------------------------------------------------

% -------------------- CHAPTER 1 -------------------------
\section{Intgeration}
\subsection{Riemann Sums and Definite Integral}
\begin{thm}[Integrability Theorem for Continuous Functions]
Let $f$ be continuous on $[a, b]$. Then $f$ is integrable on $[a,b]$. Moreover, 
$$\int^b_a f(t) \space dt = \lim_{n \to \infty}S_n$$ where, $$S_n = \sum^n_{i = 1}f(c_i) \Delta t_i$$ is any Riemann sum associated with the regular $n$-partitions. In particular, $$\int^b_a f(t) \space dt=\lim_{n \to \infty} R_n = \lim_{n \to \infty} \sum^n_{i = 1} f(t_i)\dfrac{b-a}{n}$$ and $$\int^b_a f(t) \space dt=\lim_{n \to \infty} L_n = \lim_{n \to \infty} \sum^n_{i = 1} f(t_{i - 1}) \dfrac{b-a}{n}$$
\end{thm}
\subsection{Properties of the Definite Integral}
\begin{thm}[Properties of Integrals Theorem]
Assume that $f$ and $g$ are integrable on the interval $[a, b]$. Then:
\begin{enumerate}
    \item For any $c \in \R$, $\int^b_a c \space f(t) \space dt = c \int^b_a f(t) \space dt$
    \item $\int^b_a (f + g)(t) \space dt = \int^b_a f(t) \space dt + \int^b_a g(t) \space dt$
    \item If $m \le f(t) \le M$ for all $t \in [a, b]$, then $m(b-a) \le \int^b_a f(t) \space dt \le M(b-a)$
    \item If $0 \le f(t)$ for all $t \in [a, b]$, then $0 \le \int^b_a f(t) \space dt$
    \item If $g(t) \le f(t)$ for all $t \in [a, b]$, then $\int^b_a g(t) \space dt \le \int^b_a f(t) \space dt$
    \item The function $|f|$ is integrable on $[a,b]$ and $\big|\int^b_a f(t) \space dt \big| \le \int^b_a |f(t)| \space dt$
\end{enumerate}
\end{thm}
\begin{thm}[Integrals over Subintervals Theorem]
    Assume that $f$ is integrable on an interval $I$ containing $a, b$ and $c$. Then $$\int^a_a f(t) \space dt = \int^c_a f(t) \space dt + \int^b_c f(t) \space dt$$
\end{thm}
\subsection{The Average Value of a Function}
\begin{thm}[Average Value Theorem (Mean Value Theorem for Integrals)]
    Assume that $f$ is continuous on $[a, b]$. Then there exists $a \le c \le b$ such that $$f(c) = \dfrac{1}{b-a}\int^b_af(t) \space dt$$
\end{thm}
\subsection{The Fundamental Theorem of Calculus}
\begin{thm}[Fundamental Theorem of Calculus (Part 1)]
    Assume that $f$ is continuous on an open interval $I$ containing a point $a$. Let $$G(x) =  \int^x_a f(t) dt$$Then $G(x)$ is differentiable at each $x \in I$ and $$G'(x) = f(x)$$ Equivalently, $$G'(x) = \dfrac{d}{dx}\int^x_a f(t) dt = f(x)$$
\end{thm}
\begin{thm}[Extended Version of the Fundamental Theorem of Calculus]
    Assume that $f$ is continuous and that $g$ and $h$ are differentiable. Let, $$H(x) = \int^{h(x)}_{g(x)}f(t) \space dt$$ Then $H(x)$ is differentiable and $$H'(x) = f(h(x)) h'(x) - f(g(hx))g'(x)$$
\end{thm}
\begin{thm}[Power Rule for Antiderivatives]
    If $\alpha \ne -1$, then $$\int x^\alpha \space dx = \dfrac{x^{\alpha + 1}}{\alpha + 1} + C$$
\end{thm}
\begin{thm}[Fundamental Theorem of Calculus (Part 2)]
    Assume that $f$ is continuous and that $F$ is any antiderivative of $f$, then $$\int^b_a f(t)\space dt = F(b) - F(a)$$
\end{thm}
\subsection{Change of Variables}

\begin{thm}[Change of Variables Theorem]
Assume that $g'(x)$ is continuous on $[a,b]$ and $f(u)$ is continuous on $g([a,b])$,  then $$\int^{x=b}_{x=a} f(g(x))g'(x) \space dx = \int^{u = g(b)}_{u = g(a)} f(u) \space du$$
\end{thm}




\newpage


\section{Techniques of Integration}
\subsection{Partial Fractions}
\begin{thm}[Integration by Parts Theorem]
Assume that $f$  and $g$  are such that both $f'$ and $g'$ are continuous on an interval containing $a$ and $b$. Then $$\int^b_a f(x)g'(x) \space dx = f(x)g(x)|^b_a - \int^b_a f'(x)g(x) \space dx$$
\end{thm}
\begin{thm}[Integration of Partial Fractions]
    Assume that $f(x) = \frac{p(x)}{q(x)}$ admits a Type I Partial Fraction Decomposition of the form $$f(x) = \dfrac{1}{a}\bigg[\dfrac{A_1}{x - a_1} + \dfrac{A_2}{x - a_2} + \dots + \dfrac{A_k}{x - a_k} \bigg]$$
    Then 
    \begin{align*}
        \int f(x) \space dx &= \dfrac{1}{a}\bigg[\int \dfrac{A_1}{x - a_1} \space dx + \int \dfrac{A_2}{x - a_2} \space dx + \dots + \int \dfrac{A_k}{x - a_k} \space dx \bigg]\\
        &= \dfrac{1}{a}[A_1 \ln (|x - a_1|) + A_2 \ln (|x - a_2|) + \dots + A_k \ln (|x - a_k|)] + C
    \end{align*}
\end{thm}
\subsection{Improper Integrals}
\begin{thm}[$p$-Test for Type I Improper Integrals]
    The improper integral $$\int^\infty_1 \dfrac{1}{x^p} \space dx$$ converges if and only if $p > 1$. If $p > 1$, then $$\int^\infty_1 \dfrac{1}{x^p} \space dx = \dfrac{1}{p-1}$$
\end{thm}
\begin{thm}[Properties of Type I Improper Integrals]
    Assume that $\int^\infty_a f(x) \space dx$  and $\int^\infty_a g(x) \space dx$ both converge
    \begin{enumerate}
        \item $\int^\infty_a cf(x) \space dx$ converges for each $c \in \R$ and  $$\int^\infty_a cf(x) \space dx = c \int^\infty_a f(x) \space dx$$
        \item $\int^\infty_a (f(x) + g(x)) \space dx $ converges and $$\int^\infty_a (f(x) + g(x)) \space dx  = \int^\infty_a f(x)\space dx + \int^\infty_a g(x) \space dx$$
        \item If $f(x) \le g(x) $ for all $a \le x$, then $$\int^\infty_a f(x)\space dx \le \int^\infty_a g(x) \space dx$$
        \item If $a < c< \infty $, then $\int^\infty_c f(x) \space dx$ converges and $$\int^\infty_a f(x)\space dx = \int^c_a f(x)\space dx  + \int^\infty_c f(x)\space dx$$
    \end{enumerate}
\end{thm}
\begin{thm}[The Monotone Convergence Theorem for Functions]
    Assume that $f$ is non-decreasing on $[a, \infty]$.
    \begin{enumerate}
        \item If $\{f(x) \space | \space  x \in [a, \infty]\}$ is bounded above, then $\lim_{x \to \infty}f(x)$ exists and $$\lim_{x \to \infty} f(x) = L = lub(\{f(x) \space | \space  x \in [a, \infty)\})$$
        \item If $\{f(x) \space | \space  x \in [a, \infty]\}$ is not bounded above, then $\lim_{x \to \infty} f(x) = \infty$
    \end{enumerate}
\end{thm}
\begin{thm}[Comparison Test for Type I Improper Integrals]
    Assume that $0\le g(x)\le f(x)$ forall $x\ge a$ and that both $f$ and $g$ are continuous on $[a, \infty)$.
    \begin{enumerate}
        \item If $\int^{\infty}_a f(x) \space dx $ converges, then so does $\int^\infty_a g(x) \space dx$
        \item If $\int^\infty_a g(x) \space dx$ diverges, then so does $\int^{\infty}_a f(x) \space dx$
    \end{enumerate}
\end{thm}
\begin{thm}[Absolute Convergence Theorem for Improper Integrals]
    Let $f$ be integrable on $[a,b]$ for all $b > a$. Then $|f|$ is also integrable on $[a,b]$ for all $b > a$. Moreover, if we assume that $$\int^\infty_a |f(x)|\space dx$$ converges, then so does $$\int^\infty_a f(x) \space dx$$ In particular, if $0 \le |f(x)| \le g(x)$ for all $x \ge a$, both $f$ and $g$ are integrable on $[a,b]$ for all $b \ge a$, and if  $\int^\infty_a g(x) \space dx$ converges, then so does $$\int^\infty_a f(x) \space dx$$
\end{thm}
\begin{thm}[$p$-Test for Type II Improper Integrals]
    The improper integral $$\int^1_0 \dfrac{1}{x^p} \space dx$$ converges if and only if $p < 1$. \\If $p < 1$, then $$\int^1_0 \dfrac{1}{x^p} \space dx = \dfrac{1}{1 - p}$$
\end{thm}



\newpage




\section{Applications of Integration}
\subsection{Area Between Curves}
\begin{thm}[Area Between Curves]
    Let $f$ and $g$ be continuous on $[a,b]$. Let $A$ be the region bounded by the graphs of $f$ and $g$, the line $t = a$ and the line $t = b$. Then the area of region $A$ is given by $$A = \int^b_a |g(t) - f(t)|\space dt$$
\end{thm}
\subsection{Volumes of Revolution: Disk Method}
\begin{thm}[Volumes of Revolution: Disk Method I]
    Let $f$ be continuous on $[a,b]$ with $f(x) \ge 0$  for all $x \in [a,b]$. Let $W$ be the region bounded by the graphs of $f$, the $x$-axis and the lines $x = a$ and $x = b$. Then the volume $V$ of the solid of revolution obtained by rotating the region $W$ around the $x$-axis is given by $$V = \int^b_a \pi f(x)^2 \space dx$$
\end{thm}
\begin{thm}[Volumes of Revolution: Disk Method II]
    Let $f$ and $g$ be continuous on $[a,b]$ with $0 \le f(x) \le g(x)$ for all $x \in [a,b]$. Let $W$ be the region bounded by the graphs of $f$ and $g$, and the lines $x = a$ and $x = b$. Then the volume $V$ of the solid of revolution obtained by rotating the region $W$ around the $x$-axis is given by $$V = \int^b_a \pi(g(x)^2 - f(x)^2)\space dx$$
\end{thm}
\begin{thm}[Volumes of Revolution: The Shell Method]
    Let $a \ge 0$. Let $f$ and $g$ be continuous on $ [a,b]$ with f$(x) \le g(x)$ for all $x \in [a, b]$. Let $W$ be the region bounded by the graphs of$ f$ and $g$, and the lines $x = a $ and $x = b$. Then the volume$ V$ of the solid of revolution obtained by rotating the region $W$ around the $y$-axis is given by $$V = \int^b_a 2\pi x(g(x) - f(x)) \space dx$$
\end{thm}
\subsection{Arc Length}
\begin{thm}[Arc Length]
    Let $ f $ be continuously differentiable on $[a, b]$. Then the arc length $S$ of the graph of $f$ over the interval $[a, b]$ is given by $$S = \int^b_a \sqrt{1 + (f'(x))^2} \space dx$$
\end{thm}


\newpage


\section{Differential Equations}
\subsection{First-Order Linear Differential Equations}
\begin{thm}[Solving First-order Linear Differential Equations]
    Let $f$ and $g$  be continuous and let $$y' = f(x)y + g(x)$$ be a first-order linear differential equation. Then the solutions to this equation are of the form $$y = \dfrac{\int g(x)I(x) \space dx}{I(x)}$$ where $I(x) = e^{- \int f(x) \space dx}$
\end{thm}
\subsection{Initial Value Problems}
\begin{thm}[Existence and Uniqueness Theorem for FOLDE]
    Assume that $f$ and $g$ are continuous functions on an interval $I$. Then for each $x_0 \in I$ and for all $y_0 \in \R$, the initial value problem
    \begin{align*}
        y' = f(x)y + g(x)\\
        y(x_0) = y_0
    \end{align*}
    has exactly one solution $y = \phi(x)$ on the interval $I$.
\end{thm}


\newpage


\section{Numerical Series}



\newpage




\section{Power Series}
\end{document}