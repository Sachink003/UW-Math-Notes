% header -----------------------------------------------------------------------
% Template created by texnew (author: Sachin Kumar); info can be found at 'https://github.com/alexrutar/texnew'.
% version (1.13)


% doctype ----------------------------------------------------------------------
\documentclass[11pt, a4paper]{memoir}
\usepackage[utf8]{inputenc}
\usepackage[left=3cm,right=3cm,top=3cm,bottom=4cm]{geometry}
\usepackage[protrusion=true,expansion=true]{microtype}
\usepackage[table]{xcolor}


% packages ---------------------------------------------------------------------
\usepackage{amsmath,amssymb,amsfonts}
\usepackage{graphicx}
\usepackage{etoolbox}

% Set enimitem
\usepackage{enumitem}
\SetEnumitemKey{nl}{nolistsep}
\SetEnumitemKey{r}{label=(\roman*)}

% Set tikz
\usepackage{tikz, pgfplots}
\pgfplotsset{compat=1.15}
\usetikzlibrary{intersections,positioning,cd}
\usetikzlibrary{arrows,arrows.meta}
\tikzcdset{arrow style=tikz,diagrams={>=stealth}}

% Set hyperref
\usepackage[hidelinks]{hyperref}
\usepackage{xcolor}
\newcommand\myshade{85}
\colorlet{mylinkcolor}{violet}
\colorlet{mycitecolor}{orange!50!yellow}
\colorlet{myurlcolor}{green!50!blue}

\hypersetup{
  linkcolor  = mylinkcolor!\myshade!black,
  citecolor  = mycitecolor!\myshade!black,
  urlcolor   = myurlcolor!\myshade!black,
  colorlinks = true,
}


% macros -----------------------------------------------------------------------
\DeclareMathOperator{\N}{{\mathbb{N}}}
\DeclareMathOperator{\Q}{{\mathbb{Q}}}
\DeclareMathOperator{\Z}{{\mathbb{Z}}}
\DeclareMathOperator{\R}{{\mathbb{R}}}
\DeclareMathOperator{\C}{{\mathbb{C}}}
\DeclareMathOperator{\F}{{\mathbb{F}}}

% Boldface includes math
\newcommand{\mbf}[1]{{\boldmath\bfseries #1}}

% proof implications
\newcommand{\imp}[2]{($#1\Rightarrow#2$)\hspace{0.2cm}}
\newcommand{\impe}[2]{($#1\Leftrightarrow#2$)\hspace{0.2cm}}
\newcommand{\impr}{{($\Rightarrow$)\hspace{0.2cm}}}
\newcommand{\impl}{{($\Leftarrow$)\hspace{0.2cm}}}

% align macros
\newcommand{\agspace}{\ensuremath{\phantom{--}}}
\newcommand{\agvdots}{\ensuremath{\hspace{0.16cm}\vdots}}

% convenient brackets
\newcommand{\brac}[1]{\ensuremath{\left\langle #1 \right\rangle}}
\newcommand{\norm}[1]{\ensuremath{\left\lVert#1\right\rVert}}
\newcommand{\abs}[1]{\ensuremath{\left\lvert#1\right\rvert}}

% arrows
\newcommand{\lto}[0]{\ensuremath{\longrightarrow}}
\newcommand{\fto}[1]{\ensuremath{\xrightarrow{\scriptstyle{#1}}}}
\newcommand{\hto}[0]{\ensuremath{\hookrightarrow}}
\newcommand{\mapsfrom}[0]{\mathrel{\reflectbox{\ensuremath{\mapsto}}}}
 
% Divides, Not Divides
\renewcommand{\div}{\bigm|}
\newcommand{\ndiv}{%
    \mathrel{\mkern.5mu % small adjustment
        % superimpose \nmid to \big|
        \ooalign{\hidewidth$\big|$\hidewidth\cr$/$\cr}%
    }%
}

% Convenient overline
\newcommand{\ol}[1]{\ensuremath{\overline{#1}}}

% Big \cdot
\makeatletter
\newcommand*\bigcdot{\mathpalette\bigcdot@{.5}}
\newcommand*\bigcdot@[2]{\mathbin{\vcenter{\hbox{\scalebox{#2}{$\m@th#1\bullet$}}}}}
\makeatother

% Big and small Disjoint union
\makeatletter
\providecommand*{\cupdot}{%
  \mathbin{%
    \mathpalette\@cupdot{}%
  }%
}
\newcommand*{\@cupdot}[2]{%
  \ooalign{%
    $\m@th#1\cup$\cr
    \sbox0{$#1\cup$}%
    \dimen@=\ht0 %
    \sbox0{$\m@th#1\cdot$}%
    \advance\dimen@ by -\ht0 %
    \dimen@=.5\dimen@
    \hidewidth\raise\dimen@\box0\hidewidth
  }%
}

\providecommand*{\bigcupdot}{%
  \mathop{%
    \vphantom{\bigcup}%
    \mathpalette\@bigcupdot{}%
  }%
}
\newcommand*{\@bigcupdot}[2]{%
  \ooalign{%
    $\m@th#1\bigcup$\cr
    \sbox0{$#1\bigcup$}%
    \dimen@=\ht0 %
    \advance\dimen@ by -\dp0 %
    \sbox0{\scalebox{2}{$\m@th#1\cdot$}}%
    \advance\dimen@ by -\ht0 %
    \dimen@=.5\dimen@
    \hidewidth\raise\dimen@\box0\hidewidth
  }%
}
\makeatother


% macros (theorem) -------------------------------------------------------------
\usepackage[thmmarks,amsmath,hyperref]{ntheorem}
\usepackage[capitalise,nameinlink]{cleveref}

% Numbered Statements
\theoremstyle{change}
\theoremindent\parindent
\theorembodyfont{\itshape}
\theoremheaderfont{\bfseries\boldmath}
\newtheorem{theorem}{Theorem.}[section]
\newtheorem{lemma}[theorem]{Lemma.}
\newtheorem{corollary}[theorem]{Corollary.}
\newtheorem{proposition}[theorem]{Proposition.}

% Claim environment
\theoremstyle{plain}
\theorempreskip{0.2cm}
\theorempostskip{0.2cm}
\theoremheaderfont{\scshape}
\newtheorem{claim}{Claim}
\renewcommand\theclaim{\Roman{claim}}
\AtBeginEnvironment{theorem}{\setcounter{claim}{0}}

% Un-numbered Statements
\theorempreskip{0.1cm}
\theorempostskip{0.1cm}
\theoremindent0.0cm
\theoremstyle{nonumberplain}
\theorembodyfont{\upshape}
\theoremheaderfont{\bfseries\itshape}
\newtheorem{definition}{Definition.}
\theoremheaderfont{\itshape}
\newtheorem{example}{Example.}
\newtheorem{exercise}{Exercise.}
\newtheorem{remark}{Remark.}

% Proof / solution environments
\theoremseparator{}
\theoremheaderfont{\hspace*{\parindent}\scshape}
\theoremsymbol{$//$}
\newtheorem{solution}{Sol'n}
\theoremsymbol{$\blacksquare$}
\theorempostskip{0.4cm}
\newtheorem{proof}{Proof}
\theoremsymbol{}
\newtheorem{nmproof}{Proof}

% Format references
\crefformat{equation}{(#2#1#3)}
\Crefformat{theorem}{#2Thm. #1#3}
\Crefformat{lemma}{#2Lem. #1#3}
\Crefformat{proposition}{#2Prop. #1#3}
\Crefformat{corollary}{#2Cor. #1#3}
\crefformat{theorem}{#2Theorem #1#3}
\crefformat{lemma}{#2Lemma #1#3}
\crefformat{proposition}{#2Proposition #1#3}
\crefformat{corollary}{#2Corollary #1#3}


% macros (algebra) -------------------------------------------------------------
\DeclareMathOperator{\Ann}{Ann}
\DeclareMathOperator{\Aut}{Aut}
\DeclareMathOperator{\chr}{char}
\DeclareMathOperator{\coker}{coker}
\DeclareMathOperator{\disc}{disc}
\DeclareMathOperator{\End}{End}
\DeclareMathOperator{\Fix}{Fix}
\DeclareMathOperator{\Frac}{Frac}
\DeclareMathOperator{\Gal}{Gal}
\DeclareMathOperator{\GL}{GL}
\DeclareMathOperator{\Hom}{Hom}
\DeclareMathOperator{\id}{id}
\DeclareMathOperator{\im}{im}
\DeclareMathOperator{\lcm}{lcm}
\DeclareMathOperator{\Nil}{Nil}
\DeclareMathOperator{\rank}{rank}
\DeclareMathOperator{\Res}{Res}
\DeclareMathOperator{\Spec}{Spec}
\DeclareMathOperator{\spn}{span}
\DeclareMathOperator{\Stab}{Stab}
\DeclareMathOperator{\Tor}{Tor}

% Lagrange symbol
\newcommand{\lgs}[2]{\ensuremath{\left(\frac{#1}{#2}\right)}}

% Quotient (larger in display mode)
\newcommand{\quot}[2]{\mathchoice{\left.\raisebox{0.14em}{$#1$}\middle/\raisebox{-0.14em}{$#2$}\right.}
                                 {\left.\raisebox{0.08em}{$#1$}\middle/\raisebox{-0.08em}{$#2$}\right.}
                                 {\left.\raisebox{0.03em}{$#1$}\middle/\raisebox{-0.03em}{$#2$}\right.}
                                 {\left.\raisebox{0em}{$#1$}\middle/\raisebox{0em}{$#2$}\right.}}


% macros (analysis) ------------------------------------------------------------
\DeclareMathOperator{\M}{{\mathcal{M}}}
\DeclareMathOperator{\B}{{\mathcal{B}}}
\DeclareMathOperator{\ps}{{\mathcal{P}}}
\DeclareMathOperator{\pr}{{\mathbb{P}}}
\DeclareMathOperator{\E}{{\mathbb{E}}}
\DeclareMathOperator{\supp}{supp}
\DeclareMathOperator{\sgn}{sgn}

\renewcommand{\Re}{\ensuremath{\operatorname{Re}}}
\renewcommand{\Im}{\ensuremath{\operatorname{Im}}}
\renewcommand{\d}[1]{\ensuremath{\operatorname{d}\!{#1}}}


% file-specific preamble -------------------------------------------------------
\DeclareMathOperator{\Ps}{\mathcal{P}}
\renewcommand{\Re}{\ensuremath{\operatorname{Re}}}
\renewcommand{\Im}{\ensuremath{\operatorname{Im}}}
\DeclareMathOperator{\proj}{proj}
\DeclareMathOperator{\Int}{Int}
\DeclareMathOperator{\Id}{Id}
\DeclareMathOperator{\diam}{diam}
\newcommand{\inner}[2]{\left\langle #1, #2 \right\rangle} % inner product
\newcommand{\st}{\text{ s.t. }}


% constants --------------------------------------------------------------------
\newcommand{\subject}{MATH 135 \\ Honours Algebra}
\newcommand{\semester}{Winter 2023}
\newcommand{\professor}{Jason Bell}

% formatting -------------------------------------------------------------------
% Fonts
\usepackage{kpfonts}
\usepackage{dsfont}

% Adjust numbering
\numberwithin{equation}{section}
\counterwithin{figure}{section}
\counterwithout{section}{chapter}
\counterwithin*{chapter}{part}

% Footnote
\setfootins{0.5cm}{0.5cm} % footer space above
\renewcommand*{\thefootnote}{\fnsymbol{footnote}} % footnote symbol

% Table of Contents
\renewcommand{\thechapter}{\Roman{chapter}}
\renewcommand*{\cftchaptername}{Chapter } % Place 'Chapter' before roman
\setlength\cftchapternumwidth{4em} % Add space before chapter name
\cftpagenumbersoff{chapter} % Turn off page numbers for chapter
\maxtocdepth{section} % table of contents up to section

% Section / Subsection headers
\setsecnumdepth{section} % numbering up to and including "section"
\newcommand*{\shortcenter}[1]{%
    \sethangfrom{\noindent ##1}%
    \Large\boldmath\scshape\bfseries
    \centering
\parbox{5in}{\centering #1}\par}
\setsecheadstyle{\shortcenter}
\setsubsecheadstyle{\large\scshape\boldmath\bfseries\raggedright}

% Chapter Headers
\chapterstyle{verville}

% Page Headers / Footers
\copypagestyle{myruled}{ruled} % Draw formatting from existing 'ruled' style
\makeoddhead{myruled}{}{}{\scshape\subject}
\makeevenfoot{myruled}{}{\thepage}{}
\makeoddfoot{myruled}{}{\thepage}{}
\pagestyle{myruled}
\setfootins{0.5cm}{0.5cm}
\renewcommand*{\thefootnote}{\fnsymbol{footnote}}

% Titlepage
\title{\subject}
\author{Sachin Kumar\thanks{\itshape skmuthuk@uwaterloo.ca}\\ University of Waterloo}
\date{\semester\thanks{Last updated: \today}}


%----------------------- DOCUMENT BEGIN ----------------------

\begin{document}
\pagenumbering{gobble}
\hypersetup{pageanchor=false}
\maketitle
\newpage
\frontmatter
\hypersetup{pageanchor=true}
\tableofcontents*
\newpage
\mainmatter






% main document ----------------------------------------------------------

% -------------------- CHAPTER 1 -------------------------
\chapter{Introduction to the Language of Mathematics}
\section{Sets and Mathematical Statements}
\begin{definition}[Set]
A set is a well-defined unordered (i.e., order does not matter) collection of distinct (unique) objects.\\
Example: 
Empty set = $\{\} = \emptyset$
\end{definition}
Note:
\begin{itemize}
    \item $\{\emptyset\} \ne \emptyset$ and $\{a, \{a, b\}\}$ is a set since $a, \{a, b\}$ are distinct objects.
    \item $\in \to $ "is a member of"
    \item $\notin \to $ "is not a member of"
\end{itemize}
Exercise 1 (True or false)
\begin{enumerate}
    \item $\empty \in \{A, \{A, B\}\}$ → false
    \item $A \in \{A, \{A, B\}\}$ → true
    \item $B \in \{A, \{A, B\}\}$ → false
    \item $\{B, A\} \in \{A , \{A, B\}\}$ → true
\end{enumerate}
\subsection{Common Sets}
$\N = \{1, 2, 3, \dots\}$ → Natural Numbers\\
$\Z = \{\dots, -2, -1, 0, 1, 2, \dots\}$ → Integer Numbers\\
$\R = \{\text{ set of real numbers }\}$ → Real Numbers\\
$\mathbb{P} = \{2, 3, 5, \dots \}$ → Prime Numbers\\
$\mathbb{Q} = \{\frac{a}{b}: a, b \in \Z, b \ne 0\}$ → Rational Numbers\\ \vspace{1ex} \\
Exercise 1 (True or false)
\begin{enumerate}
    \item $\mathbb{Q} \in \N$ → false
    \item $\sqrt{2} \in \mathbb{Q}$ → false
    \item $- \infty \in \Z$ → false
    \item $\pi \in \R$ → true
    \item $\{\sqrt{2}, \sqrt{3}\} \subseteq \R$  → true
\end{enumerate}
\subsection{Statements}
\begin{definition}[statements]
  A statement is a sentence that has a definite state of being true or false. (It cannot be sometimes false or sometimes true). (i.e., it cannot be sometimes false or sometimes true)  
\end{definition}
$P(a, b, c): a^2 + b^2 = c^2$ is NOT a statement since for some values of $a, b $ and $c $ the statement may be true and false for some. \\ \vspace{1ex}\\
Examples: 
\begin{itemize}
    \item $P(3, 4, 5): 3^2 + 4^2 = 5^2$  → true
    \item $P(1, 2, 3): 1^2 + 2^2 = 3^2$  → false
\end{itemize} 
Exercise 1 (Which of these are statements)
\begin{enumerate}
    \item 18 is a prime number → statement
    \item $a^3 + b^3 = c^3$  → not a statement
    \item $\forall \space a, b, c \in \R$,  $a^3 + b^3 = c^3$  → statement
    \item $\exists \space a, b , c \in \R$ such that $a^3 + b^3 = c^3$ → statement
\end{enumerate}
\subsection{Open Sentence}
\begin{definition}[open sentence]
    An open sentence is a sentence with at least one variable that is not a statement but can become one when we give values.
\end{definition}
$P(a, b, c): a^3 + b^3 = c^3$ is an open sentence since if we give values for $a, b $ and $c$ $P(a,b , c) $ will become a statement. 
\\ \vspace{1ex} \\
Examples:
\begin{itemize}
    \item $P(2, 3 ,5): 2^3 + 3^3 = 5^3$ → statement
    \item $\forall \space a, b, c \in \R$,  $a^3 + b^3 = c^3$  → statement
\end{itemize} 
Exercise 4 (Which of these are open sentences or statements)
\begin{enumerate}
    \item $P(2, 3, 5)$ → statement
    \item $\forall a \in \R,$   $P(a, b , c)$ → open sentence
    \item $P(2, 3, C)$ → open sentence
    \item $\exists a, b \in \N$ and $c \in \R$ such that $P(a, b, c)$ → sometimes considered as a statement or as an open sentence
\end{enumerate}
\subsection{Negation of Statements}
If $P$ is a statement, $$\neg P \to \text{not }P$$
\\ \; \\
Examples:
\begin{itemize}
    \item $P \to$  There is a $x \in \R$ such that $x^2 = 2$
    \item $\neg P \to$  $\forall \space x \in \R,$   $x^2 \ne 2$
    \item $P \to$  $\forall \space x \in \R$,   $x^2 = 2$
    \item $\neg P \to$  $\exists \space x \in \R$ such that $x^2 = 2$
\end{itemize}
$\neg (\neg P)$ is logically equivalent to $P$, $\neg (\neg P) = P$

\section{Quantifiers and Nested Quantifiers}
\subsection{Quantifiers}
$\forall$ = "for all" $\to$ universal quantifier\\
$\exists$ = "there exists" $\to$ Existential quantifier\\
Typically , we'll have a open sentence with at least one "free" variable, $x$. 
\\ \vspace{1ex} \\
Examples: 
\begin{itemize}
    \item $x^2 + 2 = z^3$
    \item $\forall x \in \Z, \exists z \in \R, x^2 + 2 = z^3 \to $ For all $x \in \Z$, there exists $z \in \R$, such that $x^2 + 2 = z^3$
\end{itemize}
P(x): $x^2 = 2$ (open sentence)
\begin{itemize}
    \item $\exists x \in \N, x^2 = 2 \to $ false
    \item $\exists x \in \R, x^2 = 2 \to $ true
    \item $\forall x \in \R, x^2 \ne 2 \to $ true
    \item $\forall x \in \R, x^2 \ne 2 \to $ false
\end{itemize}
If Q(x): $\frac{m+1}{m+2} = 5$ is open sentence, then \\
$$\exists m \in \Z, \frac{m+1}{m+2} = 5 \to \text{ false }$$\\
We can make the above statement true by changing its domains, i.e., \\
\begin{align*}
    &\exists m \in \R, \frac{m+1}{m+2} = 5
    &\exists m \in \Q, \frac{m+1}{m+2} = 5
\end{align*}
\subsection{Hidden Quantifiers}
Examples:
\begin{itemize}
    \item $64$ is a perfect square $\to \exists x \in \Z, x^2 = 64$ (true statement)
    \item $2^{2x - 4} = 8$ has a integer solution $\to \exists x \in \Z, 2^{2x - 4} = 8$ (false statement)
    \item The graph of $y = x^3 - 2x + 1$ has no $x$-intercept\\ $\to $ There is no solution in $x \in \R$ such that $x^3 - 2x + 1 = 0$ \\ $\to$ For all $x \in \R, x^3 - 2x + 1 = 0$ \\$\to \forall x \in \R, x^3 - 2x + 1 = 0$ (false statement)
\end{itemize}
\subsection{Negation of Quantifiers}
$P: $ Everyone in this room was born in or before 2013. \\
$\neg P: $ There exists someone in this room was born in or after 2013.\\
$S: $ Set of people in this room, $Q(x) = x$ is born in or before 2013, where $x$ is a person in the room.
\begin{itemize}
    \item $P: \forall x \in S, Q(x)$
    \item $\neg P: \exists x \in S, \neg Q(x)$
\end{itemize}
Fact: If we have the statement of the form 
\begin{align*}
    P&: \forall x \in S, Q(x)\\
    \neg P&: \exists x \in S, \neg Q(x)
\end{align*}
\\ \vspace{1ex} \\
Exercise 5 (Negate the statement)
\begin{enumerate}
    \item $P: \forall x \in \R, |x| \ge 5$
    \item $\neg P: \exists x \in \R, \neg(|x| \ge 5) \to \exists x \in \R, |x| < 5$
\end{enumerate}
\subsection{Nested Quantifiers}
Examples:
Let $Q(x, y) = x^3 - y^3 = 1$
\begin{itemize}
    \item $\forall x \in \R, \forall y \in \R, x^3 - y^3 = 1 \to $ false
    \item $\exists x \in \R, \exists y \in \R, x^3 - y^3 \to $ true
    \item $\forall x \in \R, \exists y \in \R, x^3 - y^3 \to $ true
    \item $\exists x \in \R, \forall y \in \R, x^3 - y^3 \to $ false
\end{itemize}
Note: Switching the order of the quantifiers in a statement makes a difference.\\
If we have an open sentence $Q(x, y)$,\\
$\exists x \in S, \forall y \in T, Q(x, y) \to $ There is an $x \in S$ such that [for all $y \in T, Q(x,y)$] is true.

\subsection{Negating Nested Quantifiers}
Examples: 
Let $Q(x, y, z) = x^5 + y^2 = 2^3$\\
\begin{align*}
    P&: \exists x \in \Z, \forall y \in \Q, \, \exists z \in \R, \, x^5 + y^2 = 2^3\\
    \neg P&: \forall x \in \Z, \exists y \in \Q, \, \forall z \in \R, \, x^5 + y^2 \ne 2^3
\end{align*}
Fact: 
\begin{itemize}
    \item In order to negate a nested quantified statement, just flip $\forall$ and $\exists$, and also negate the statement $P(x)$
    \item Also if the nested quantified statement is long, break it into shorter nested quantified statements and negate it.
\end{itemize}


\chapter{Logical Analysis of Mathematical Statements}
\section{Truth Table}
Let $p$ be statement. 
\begin{enumerate}
    \item $\neg P$ = "not $P$"
    \item $\neg P$ is true when $P$ is false and false when $P$ is true.
\end{enumerate}

\begin{table}[ht]
\centering
\begin{tabular}{|c|c|c|}
    \hline
    $P$ & $\neg P$ & $\neg (\neg P)$ \\
    \hline
    T & F & T \\
    \hline
    F & T & F \\
    \hline
\end{tabular}
\end{table}

\begin{itemize}
    \item $\neg$ is a logical operator, somethin take takes a statement and creates a new statements.
    \item $\neg P$ is a logical expression.
\end{itemize}
\textbf{Notice:} $P$ and $\neg(\neg P)$ have the same truth value So $P$ and $\neg (\neg P)$ are logically equivalent ($\equiv$), i.e, 
$$P \equiv \neg (\neg P)$$ 

\section{Conjuction and Disjunction}
\begin{enumerate}
    \item Conjuction ($\wedge$) = "and"
    \item Disjunction ($\vee$) = "or"
\end{enumerate}
We can use conjunction and disjunction to create compound statements, that are built from two or more statement using
things like $\vee$ and $\wedge$. \\ \: \\
\textbf{Example:} $A$ and $B$ are statements (statement variables)

\begin{table}[ht]
    \centering
    \begin{tabular}{|c|c|c|c|}
        \hline
        $A$ & $B$ & $A \wedge B$ & $A \vee B$ \\
        \hline
        T & T & T & T \\
        \hline
        T & F & F & T \\
        \hline
        F & T & F & T \\
        \hline 
        F & F & F & F \\ 
        \hline
    \end{tabular}
    \end{table}

\begin{itemize}
    \item $A \wedge B$ and $A \vee B$ are called compound statements.
    \item $\vee \to $ If one of the statements are true or both of them are, then the compound statement is true.
    \item $\wedge \to $ Even if one of the statement is false, then the compound statement is false.
\end{itemize}
\textbf{Example:} $\forall x \in \R, (x^2 \ge 0) \wedge (\sin^2(x) + \cos^2(x) = 1)$ is a true statement because both the statement variables are true. Therefore the compound statement is true.

\section{Logical Operators and Algebra}
\subsection{De Morgan's Laws}

\begin{table}[ht]
    \centering
    \begin{tabular}{|c|c|c|c|c|c|}
        \hline
        $A$ & $B$ & $\neg (A \wedge B)$ & $\neg (A \vee B)$ & $\neg A \vee \neg B$ & $\neg A \wedge \neg B$\\
        \hline
        T & T & F & F & F & F\\
        \hline
        T & F & T & F & T & F\\
        \hline
        F & T & T & F & T & F\\
        \hline 
        F & F & T & T & T & T\\ 
        \hline
    \end{tabular}
    \end{table}

\begin{enumerate}
    \item $\neg (A \wedge B) \equiv (\neg A) \vee (\neg B)$
    \item $\neg (A \vee B) \equiv (\neg A) \wedge (\neg B)$
\end{enumerate}
\textbf{Exercise:} Negate the statement\\ 
Let $L$ be a line and $P$ be a parabola.
Statement: The point $(1, 2)$ lies on $L$ or on $P$ $$(1,2) \in L \vee (1,2) \in P$$
\textbf{Sol}:
\begin{align*}
    &\neg ((1,2) \in L \vee (1,2) \in P)\\
    \equiv &(\neg (1,2) \in L \wedge \neg (1,2) \in P)\\
    \equiv &(1,2) \notin L \wedge (1,2) \notin P
\end{align*}
Negated statement: The point $(1,2)$ does not lie on $L$ and does not lie on $P$.

\subsection{Other Logical Operators Laws}
\textbf{Commutative Laws:} (order does not matter)
\begin{itemize}
    \item $P \wedge Q \equiv Q \wedge P$
    \item $P \vee Q \equiv Q \vee P$
\end{itemize}
\textbf{Associative Laws:} (parentheses does not matter)
\begin{itemize}
    \item $(P \wedge Q) \wedge R \equiv P \wedge (Q \wedge R)$
    \item $(P \vee Q) \vee R \equiv P \vee (Q \vee R)$
\end{itemize}
\textbf{Distributive Laws:}
\begin{itemize}
    \item $P \wedge (Q \vee R) \equiv (P \wedge Q) \vee (P \wedge R)$
    \item $P \vee (Q \wedge R) \equiv (P \vee Q) \wedge (P \vee R)$
\end{itemize}
\textbf{Exercise:} 
\begin{enumerate}
    \item Prove without using the truth table $$\neg (A \wedge (\neg B \wedge C)) \equiv \neg (A \wedge C) \vee B$$
    \begin{proof}
        Lets consider the LHS, 
        \begin{align*}
            \neg &(A \wedge (\neg B \wedge C))\\
            \equiv &\neg A \vee (\neg (B \wedge C)) \; \; \; \; \; \; \; \; \;[\text{By De Morgan's Law}] \\
            \equiv &\neg A \vee (\neg (\neg B) \vee \neg C) \; \; \; [\text{By De Morgan's Law}]\\
            \equiv &\neg A \vee (B \vee \neg C) \; \; \; \; \; \; \; \; \; \; \; \;[\text{By Double Negation}]\\
            \equiv &(\neg A \vee \neg C) \vee B \; \; \; \; \; \; \; \; \; \; \; \;[\text{By Associative Law}]\\
            \equiv &\neg (A \wedge C) \vee B \; \; \; \; \; \; \; \; \; \; \; \;\; \; \;[\text{By De Morgan's Law}]
        \end{align*}
    \end{proof}
    \item True or false statement: 
    \begin{enumerate}
        \item $\forall x \in \emptyset, x^2 = 1 \to$ vacuously true
        \item $\exists x \in \emptyset, x^2 = 1 \to$ false
    \end{enumerate}
\end{enumerate}

\section{Implications}
$P$ and $Q$ are statements $\to$ If $P$, then $Q$\\
"If the statement $P$ is true, then the statement $Q$ is true"\\ \: \\
\textbf{Exercise:}
\begin{enumerate}
    \item If Alice is from Canada, then Alice is from North America $\to$ True
    \item If Alice if from North America, then Alice is from Canada $\to$ False
    \item If I am an animal, then I will give you \${10} $\to$ True
    \item If $x > 3$, then $x > 5 \to$ False
    \item If $x > 3$, then $x \ge 1 \to$ True 
\end{enumerate}
\textbf{Note:} If $P$ is false, then it does not matter, if $Q$ is false or true, the implication will be \textbf{true}.
% \begin{table}[ht]
%     \centering
%     \begin{tabular}{|c|c|c|c|}
%         \hline
%         $P$ & $Q$ & $(P \implies Q)$ & $(\neg P \vee B)$ \\
%         \hline
%         T & T & T & T \\
%         \hline
%         T & F & F & F \\
%         \hline
%         F & T & T & T \\
%         \hline 
%         F & F & T & T \\ 
%         \hline
%     \end{tabular}
% \end{table}
\textbf{Implication Law: } $(P \implies Q) \equiv (\neg P \vee B)$\\ \: \\
\textbf{Exercise:}
\begin{enumerate}
    \item Prove using the implication law
    $$\forall x \in \R, (x > 2) \implies (x^2 > 1)$$
    \begin{proof}
        Let $A(x) = x > 2$ and $B(x) = x^2 > 1$. Then for $(A(x) \implies B(x))$ to be true. We need $B(x)$ to be true or $A(x)$ to be false. Notice 
        $B(x)$ is true for $x \in (1, \infty) \cup (-\infty, -1)$ and $A(x)$ is false for $x \in (-\infty, 2]$. So
        $B(x)$ is true or $A(x)$ is false, holds for $x \in (1, \infty) \cup (-\infty, -1) \cup (-\infty, 2] = \R$.\\
        So, $\forall x \in \R, B(x) \vee \neg A(x) \equiv \forall x \in \R, A(x) \implies B(x)$ 
    \end{proof}
    \item Let $\mathbb{P}$ be a set of prime numbers. Prove that $$\forall p \in \mathbb{P}, (p > 2) \implies (P + 1) \text{ is even.}$$
    \begin{proof}
        If the hypothesis is true, then $p$ is a prime greater than 2 and since $p$ it
        prime it cannot be a multiple of 2, so by definition of even and odd, $p$ is odd. so $p + 1$ is even.
        Thus if the hypothesis is true, the conclusion is true. so $\forall p \in \mathbb{P}, (p > 2) \implies (P + 1)$ is even.
    \end{proof}
\end{enumerate}
\subsection{Negating Implication}
What is $\neg (A \implies B)$?
\begin{align*}
    \neg (A \implies B) &\equiv \neg (B \vee (\neg A)) \; \; \; \; [\text{we know }(A \implies B) \equiv (B \vee (\neg A))]\\
    &\equiv \neg B \wedge  \neg (\neg A) \; \; \; \; [\text{By De Morgan's Law}]\\
    &\equiv \neg B \wedge A \; \; \; \; \; \; \; \; \; \; \; \; \;[\text{By Double Negation}]
\end{align*}
\textbf{Exercise: } If $\mathbb{P}$ are the set of prime numbers, then "There is at most one prime number less than 3".
$$\forall x \in \mathbb{P}, \forall y \in \mathbb{P}, ((x < 3) \wedge (y < 3) \implies (x = y))$$

\section{Converse and Contrapositive}
\begin{definition}[Converse]
    The implication $B \implies A$ is called the \textbf{converse} of $A \implies B$
\end{definition}
\textbf{Note: }A common mistake is to think that the implication $A \implies A$ and its converse $B \implies A$ are logically equivalent. They are not!
\begin{definition}[contrapositive]
    The implication $(\neg B) \implies (\neg A)$ is called the \textbf{contrapositive} of $A \implies B$.
\end{definition}
\textbf{Contrpositive equivalence Law:} An Implication is logically equivalent to its contrapositive. $$(P \implies Q) \equiv ((\neg B) \implies (\neg A))$$
\textbf{Implication Law:} $$(A \implies B) \equiv ((\neg A) \vee B) \hspace{7ex} \neg(A \implies B) \equiv (A \wedge (\neg B))$$

\section{If and Only if}
\begin{definition}[if and only if (iff)]
    The truth value for “$A$ if and only if $B$", written symbolically as $A \iff B$ is true when $A$ and $B$ have the 
    same truth values, and is false when they have opposite truth values.
\end{definition}
\textbf{More Laws:} 
\begin{enumerate}
    \item $(A \iff B) \equiv ((A \implies B) \wedge (B \implies A))$
    \item $\big(\forall x \in X, P(x) \iff Q(x)\big) \equiv \big((\forall x \in X, P(x) \implies Q(x)) \wedge (\forall x \in X, Q(x) \implies P(x))\big)$
\end{enumerate}
\end{document}
