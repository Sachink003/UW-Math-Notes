\documentclass[10pt]{article} 

\usepackage{fullpage}
\usepackage{bookmark}
\usepackage{amsmath}
\usepackage{amssymb}
\usepackage[dvipsnames]{xcolor}
\usepackage{hyperref} % for the URL
\usepackage[shortlabels]{enumitem}
\usepackage{mathtools}
\usepackage[most]{tcolorbox}
\usepackage[amsmath,standard,thmmarks]{ntheorem} 
\usepackage{physics}
\usepackage{pst-tree} % for the trees
\usepackage{verbatim} % for comments, for version control
\usepackage{tabu}
\usepackage{tikz}
\usepackage{float}
\usepackage{etoolbox}

\lstnewenvironment{python}{
\lstset{frame=tb,
language=Python,
aboveskip=3mm,
belowskip=3mm,
showstringspaces=false,
columns=flexible,
basicstyle={\small\ttfamily},
numbers=none,
numberstyle=\tiny\color{Green},
keywordstyle=\color{Violet},
commentstyle=\color{Gray},
stringstyle=\color{Brown},
breaklines=true,
breakatwhitespace=true,
tabsize=2}
}
{}

\lstnewenvironment{cpp}{
\lstset{
backgroundcolor=\color{white!90!NavyBlue},   % choose the background color; you must add \usepackage{color} or \usepackage{xcolor}; should come as last argument
basicstyle={\scriptsize\ttfamily},        % the size of the fonts that are used for the code
breakatwhitespace=false,         % sets if automatic breaks should only happen at whitespace
breaklines=true,                 % sets automatic line breaking
captionpos=b,                    % sets the caption-position to bottom
commentstyle=\color{Gray},    % comment style
deletekeywords={...},            % if you want to delete keywords from the given language
escapeinside={\%*}{*)},          % if you want to add LaTeX within your code
extendedchars=true,              % lets you use non-ASCII characters; for 8-bits encodings only, does not work with UTF-8
% firstnumber=1000,                % start line enumeration with line 1000
frame=single,	                   % adds a frame around the code
keepspaces=true,                 % keeps spaces in text, useful for keeping indentation of code (possibly needs columns=flexible)
keywordstyle=\color{Cyan},       % keyword style
language=c++,                 % the language of the code
morekeywords={*,...},            % if you want to add more keywords to the set
% numbers=left,                    % where to put the line-numbers; possible values are (none, left, right)
% numbersep=5pt,                   % how far the line-numbers are from the code
% numberstyle=\tiny\color{Green}, % the style that is used for the line-numbers
rulecolor=\color{black},         % if not set, the frame-color may be changed on line-breaks within not-black text (e.g. comments (green here))
showspaces=false,                % show spaces everywhere adding particular underscores; it overrides 'showstringspaces'
showstringspaces=false,          % underline spaces within strings only
showtabs=false,                  % show tabs within strings adding particular underscores
stepnumber=2,                    % the step between two line-numbers. If it's 1, each line will be numbered
stringstyle=\color{GoldenRod},     % string literal style
tabsize=2,	                   % sets default tabsize to 2 spaces
title=\lstname}                   % show the filename of files included with \lstinputlisting; also try caption instead of title
}
{}

% floor, ceiling, set
\DeclarePairedDelimiter{\ceil}{\lceil}{\rceil}
\DeclarePairedDelimiter{\floor}{\lfloor}{\rfloor}
\DeclarePairedDelimiter{\set}{\lbrace}{\rbrace}
\DeclarePairedDelimiter{\iprod}{\langle}{\rangle}

\DeclareMathOperator{\Int}{int}
\DeclareMathOperator{\mean}{mean}

% commonly used sets
\DeclareMathOperator{\N}{{\mathbb{N}}}
\DeclareMathOperator{\Q}{{\mathbb{Q}}}
\DeclareMathOperator{\Z}{{\mathbb{Z}}}
\DeclareMathOperator{\R}{{\mathbb{R}}}
\DeclareMathOperator{\C}{{\mathbb{C}}}
\DeclareMathOperator{\F}{{\mathbb{F}}}

\newcommand{\mbf}[1]{{\boldmath\bfseries #1}}

% proof implications
\newcommand{\imp}[2]{($#1\Rightarrow#2$)\hspace{0.2cm}}
\newcommand{\impe}[2]{($#1\Leftrightarrow#2$)\hspace{0.2cm}}
\newcommand{\impr}{{($\Rightarrow$)\hspace{0.2cm}}}
\newcommand{\impl}{{($\Leftarrow$)\hspace{0.2cm}}}

% align macros
\newcommand{\agspace}{\ensuremath{\phantom{--}}}
\newcommand{\agvdots}{\ensuremath{\hspace{0.16cm}\vdots}}

% convenient brackets
\newcommand{\brac}[1]{\ensuremath{\left\langle #1 \right\rangle}}

% arrows
\newcommand{\lto}[0]{\ensuremath{\longrightarrow}}
\newcommand{\fto}[1]{\ensuremath{\xrightarrow{\scriptstyle{#1}}}}
\newcommand{\hto}[0]{\ensuremath{\hookrightarrow}}
\newcommand{\mapsfrom}[0]{\mathrel{\reflectbox{\ensuremath{\mapsto}}}}

\DeclareMathOperator{\Ann}{Ann}
\DeclareMathOperator{\Aut}{Aut}
\DeclareMathOperator{\chr}{char}
\DeclareMathOperator{\coker}{coker}
\DeclareMathOperator{\disc}{disc}
\DeclareMathOperator{\End}{End}
\DeclareMathOperator{\Fix}{Fix}
\DeclareMathOperator{\Frac}{Frac}
\DeclareMathOperator{\Gal}{Gal}
\DeclareMathOperator{\GL}{GL}
\DeclareMathOperator{\Hom}{Hom}
\DeclareMathOperator{\id}{id}
\DeclareMathOperator{\im}{im}
\DeclareMathOperator{\lcm}{lcm}
\DeclareMathOperator{\Nil}{Nil}
\DeclareMathOperator{\Spec}{Spec}
\DeclareMathOperator{\spn}{span}
\DeclareMathOperator{\Stab}{Stab}
\DeclareMathOperator{\Tor}{Tor}

\newcommand{\sset}{\subseteq}

\theoremstyle{break}
\theorembodyfont{\upshape}

\newtheorem{thm}{Theorem}[subsection]
\tcolorboxenvironment{thm}{
enhanced jigsaw,
colframe=Dandelion,
colback=White!90!Dandelion,
drop fuzzy shadow east,
rightrule=2mm,
sharp corners,
before skip=10pt,after skip=10pt
}

\newtheorem{cor}{Corollary}[thm]
\tcolorboxenvironment{cor}{
boxrule=0pt,
boxsep=0pt,
colback={White!90!RoyalPurple},
enhanced jigsaw,
borderline west={2pt}{0pt}{RoyalPurple},
sharp corners,
before skip=10pt,
after skip=10pt,
breakable
}

\newtheorem{pt}[thm]{Proof Technique}
\tcolorboxenvironment{pt}{
enhanced jigsaw,
colframe=Red,
colback={White!95!Red},
rightrule=2mm,
sharp corners,
before skip=10pt,after skip=10pt
}

\newtheorem{ex}[thm]{Example}
\tcolorboxenvironment{ex}{% from ntheorem
blanker,left=5mm,
sharp corners,
before skip=10pt,after skip=10pt,
borderline west={2pt}{0pt}{Gray}
}

\newtheorem*{pf}{Proof}
\tcolorboxenvironment{pf}{% from ntheorem
breakable,blanker,left=5mm,
sharp corners,
before skip=10pt,after skip=10pt,
borderline west={2pt}{0pt}{NavyBlue!80!white}
}

\newtheorem{defn}{Definition}[subsection]
\tcolorboxenvironment{defn}{
enhanced jigsaw,
colframe=Cerulean,
colback=White!90!Cerulean,
drop fuzzy shadow east,
rightrule=2mm,
sharp corners,
before skip=10pt,after skip=10pt
}

\newtheorem{prop}[thm]{Proposition}
\tcolorboxenvironment{prop}{
boxrule=0pt,
boxsep=0pt,
colback={White!90!Green},
enhanced jigsaw,
borderline west={2pt}{0pt}{Green},
sharp corners,
before skip=10pt,
after skip=10pt,
breakable
}

\setlength\parindent{0pt}
\setlength{\parskip}{2pt}

\newcommand{\subject}{MATH 135 \\ Proof Techniques}
\newcommand{\semester}{Winter 2023}
\newcommand{\professor}{Jason Bell}

\begin{document}
\let\ref\Cref

\title{\subject}
\author{Sachin Kumar\thanks{\itshape skmuthuk@uwaterloo.ca}\\ University of Waterloo}
\date{\semester\thanks{Last updated: \today}}

\maketitle
\newpage
\tableofcontents
% \listoffigures
% \listoftables
\newpage






% main document ----------------------------------------------------------

% -------------------- CHAPTER 1 -------------------------
\section{Proving Mathematical Statements}
\subsection{Proving Universally Quantified Statements}
\begin{pt}
    To prove the universally quantified statement "$\forall x \in S, P(x)$": 
    \vspace{0.5ex} \\ 
    Choose a representative mathematical object $x \in S$. This cannot be a specific object. 
    It has to be a placeholder, that is, a variable, so that our argument would work for any 
    specific member of the domain $S$.\\
    Then, show that the open sentence $P$ must be true for our representative $x$, using known 
    facts about the elements of $S$.
\end{pt}
\begin{pt}
    To disprove the universally quantified statement "$\forall x \in S, P(x)$": \\
    Find an element $x \in S$ for which the open sentence $P(x)$ is false. This process is called finding a \emph{\textbf{counter-example}}.
\end{pt}

\subsection{Proving Existentially Quantified Statements}
\begin{pt}
    To prove the existentially quantified statement "$\exists x \in S, P(x)$": \\
    Provide an explicit value of $x$ from the domain $S$, and show that $P(x)$
    is true for this value of $x$. In other words, find an element of $S$ that 
    satisfies property $P$.
\end{pt}
\begin{pt}
    To disprove the existentially quantified statement "$\exists x \in S, P(x)$": \\
    Prove the universally quantified statement "$\forall x \in S, \neg P(x)$".
\end{pt}

\subsection{Proving Implications}
\begin{pt}
    For proving an implication: 
    \begin{enumerate}
    \item To prove the implication "$A \implies B$", \textbf{assume} that the hypothesis 
    $A$ is true, and use this assumption to show that the conclusion $B$ is true. The 
    hypothesis $A$ is what you start with. The conclusion $B$ is where you must end up.
    \item To prove the universally quantified implication "$\forall x \in S, P(x) \implies Q(x)$":\\
    Let $x$ be an arbitrary element of $S$, assume that the hypothesis $P(x)$ is true, and 
    use this assumption to show that the conclusion $Q(x)$ is true.
\end{enumerate}
\end{pt}

\subsection{Proof by Contrapositive}
\begin{pt}
    For proving an implication using the contrapositive:
\begin{enumerate}
    \item To prove the implication "$A \implies B$", replace it with its contrapositive
    "$(\neg B) \implies (\neg A)$". Then prove this contrapositive, usually by a direct 
    proof. That is, assume $\neg B$ is true and deduce that $\neg A$ must be true as well.
    \item To prove the universally quantified implication 
    "$\forall x \in S, P(x) \implies Q(x)$", replace it with its universally quantified contrapositive
    "$\forall x \in S, (\neg Q(x)) \implies (\neg P(x))$". Then prove this universally quantified contrapositive.
\end{enumerate}
\end{pt}

\subsection{Proof by Contradiction}
\begin{pt}
To prove $P(x)$ is true. Assume $\neg P(x)$ is true and come to a conclusion that $\neg P(x)$ is false, which proves that $P(x)$ is true.
\end{pt}
    
\subsubsection{Proving Uniqueness}
\begin{pt}
    To prove the statement “There is a unique element $x \in S$ such that $P(x)$ is true”:
    \begin{enumerate}
        \item[1.] (“Existence”): Prove that there is at least one element $x \in S$ such that $P(x)$ is true (ie., prove the existentially quantified statement "$\exists$ $x \in S$, $P(x)$").
        \item[2.] ("Uniqueness"): Do either (a) or (b) below.
            \begin{enumerate}
                \item Assume that $P(x)$ and $P(y)$ are true for $x, y \in S$, and prove that this assumption leads to the conclusion $x = y$,
                \item Assume that $P(x)$ and $P(y)$ are true for distinct $x, y \in S$ (so $x \ne y$), and prove that this assumption leads to a contradiction.
            \end{enumerate}
    \end{enumerate}
\end{pt}

\subsection{Proving If and Only if}
\begin{pt}
    For proving an if and only if statement:
    \begin{enumerate}
        \item To prove the statement $"A \iff B"$, it is equivalent to prove both the implication $"A \implies B"$ and its converse $"B \implies A"$.
        \item To prove the universally quantified statement $"\forall$ $x \in S$, $P(x) \iff Q(x)$", it is equivalent to do either (a) or (b) below.
        \begin{enumerate}
            \item Let $x$ be an arbitrary element of $S$, and prove both the implication $"P(x) \implies Q(x)"$ and its converse $"Q(x) \implies P(x)"$, 
            \item Prove both the universally quantified implication $"\forall$ $x \in S$, $P(x) \implies Q(x)"$ and its universally quantified converse $"\forall$ $x \in S$, $Q(x) \implies P(x)"$.
        \end{enumerate} 
    \end{enumerate}
\end{pt}

\newpage

\section{Mathematical Induction and Sets}
\subsection{Proof by Induction}
\begin{pt}
    To prove the universally quantified statement "$\forall$ $n \in \N, P(n)$":
    \begin{enumerate}
        \item Prove "$P(1)$".
        \item Prove the universally quantified implication "$\forall$ $k \in \N$, $P(k) \implies P(k+1)$".
    \end{enumerate}
    where, $P(1)$ is the \emph{base case/step}, $P(k)$ is the \emph{inductive hypothesis} and $P(k+1)$ is the \emph{inductive conclusion} and together $P(k) \implies P(k+1)$ is the \emph{inductive step}. 
\end{pt}
\subsection{Proof by Strong Induction}
\begin{pt}
    To prove the universally quantified statement “$\forall$ $n \in \N, n \ge b, P(n)$":
    \begin{enumerate}
        \item Prove "$P(b) \wedge P(b+1) \wedge \dots \wedge P(B)$", for some integer $B \ge b$.
        \item Prove the universally quantified implication "$\forall$ $k \in \Z, k \ge B$, $P(b) \wedge P(b+1) \wedge \dots \wedge P(k) \implies P(k+1)$".
    \end{enumerate}
    To implement the proof method of strong induction, what changes are needed in terms of the standard format?
    \begin{itemize}
        \item For the base case, prove all of $P(b)$, $P(b+1)$, \dots, $P(B)$. When $b < B$ there is more than one case to prove, so we label them as Base cases, and refer to $b$
        as the smallest base case and $B$ as the largest base case.
        \item For the inductive step, we assume that $k$ is an arbitrary integer where $k \ge B$. We assume the inductive hypothesis, $P(b) \wedge P(b+1) \wedge \dots \wedge P(k)$. That is, we assume $P(i)$, for all integers $i = b, b+1, \dots, k$. We then prove $P(k+1)$ using the assumption
        $P(i)$ for all integers $i = b, b + 1, \dots, k$.
    \end{itemize}
\end{pt}
\subsection{Proving subset}
\begin{pt}
    To prove that $S \subseteq T$, prove the universally quantified implication: 
    \begin{align*}
        \forall x \in \mathcal{U}, (x \in S) \implies (x \in T)
    \end{align*}
\end{pt}
\subsection{Proving equality}
\begin{pt}
    To prove that $S = T$, prove $S \subseteq T$ and $T \subseteq S$ (ie., universally qunatified if and only if): 
    \begin{align*}
        \forall x \in \mathcal{U}, (x \in S) \iff (x \in T)
    \end{align*}
\end{pt}
\end{document}
