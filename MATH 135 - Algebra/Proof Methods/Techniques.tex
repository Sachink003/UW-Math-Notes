\documentclass[10pt]{article} 

\usepackage{fullpage}
\usepackage{bookmark}
\usepackage{amsmath}
\usepackage{amssymb}
\usepackage[dvipsnames]{xcolor}
\usepackage{hyperref} % for the URL
\usepackage[shortlabels]{enumitem}
\usepackage{mathtools}
\usepackage[most]{tcolorbox}
\usepackage[amsmath,standard,thmmarks]{ntheorem} 
\usepackage{physics}
\usepackage{pst-tree} % for the trees
\usepackage{verbatim} % for comments, for version control
\usepackage{tabu}
\usepackage{tikz}
\usepackage{float}
\usepackage{etoolbox}

\lstnewenvironment{python}{
\lstset{frame=tb,
language=Python,
aboveskip=3mm,
belowskip=3mm,
showstringspaces=false,
columns=flexible,
basicstyle={\small\ttfamily},
numbers=none,
numberstyle=\tiny\color{Green},
keywordstyle=\color{Violet},
commentstyle=\color{Gray},
stringstyle=\color{Brown},
breaklines=true,
breakatwhitespace=true,
tabsize=2}
}
{}

\lstnewenvironment{cpp}{
\lstset{
backgroundcolor=\color{white!90!NavyBlue},   % choose the background color; you must add \usepackage{color} or \usepackage{xcolor}; should come as last argument
basicstyle={\scriptsize\ttfamily},        % the size of the fonts that are used for the code
breakatwhitespace=false,         % sets if automatic breaks should only happen at whitespace
breaklines=true,                 % sets automatic line breaking
captionpos=b,                    % sets the caption-position to bottom
commentstyle=\color{Gray},    % comment style
deletekeywords={...},            % if you want to delete keywords from the given language
escapeinside={\%*}{*)},          % if you want to add LaTeX within your code
extendedchars=true,              % lets you use non-ASCII characters; for 8-bits encodings only, does not work with UTF-8
% firstnumber=1000,                % start line enumeration with line 1000
frame=single,	                   % adds a frame around the code
keepspaces=true,                 % keeps spaces in text, useful for keeping indentation of code (possibly needs columns=flexible)
keywordstyle=\color{Cyan},       % keyword style
language=c++,                 % the language of the code
morekeywords={*,...},            % if you want to add more keywords to the set
% numbers=left,                    % where to put the line-numbers; possible values are (none, left, right)
% numbersep=5pt,                   % how far the line-numbers are from the code
% numberstyle=\tiny\color{Green}, % the style that is used for the line-numbers
rulecolor=\color{black},         % if not set, the frame-color may be changed on line-breaks within not-black text (e.g. comments (green here))
showspaces=false,                % show spaces everywhere adding particular underscores; it overrides 'showstringspaces'
showstringspaces=false,          % underline spaces within strings only
showtabs=false,                  % show tabs within strings adding particular underscores
stepnumber=2,                    % the step between two line-numbers. If it's 1, each line will be numbered
stringstyle=\color{GoldenRod},     % string literal style
tabsize=2,	                   % sets default tabsize to 2 spaces
title=\lstname}                   % show the filename of files included with \lstinputlisting; also try caption instead of title
}
{}

% floor, ceiling, set
\DeclarePairedDelimiter{\ceil}{\lceil}{\rceil}
\DeclarePairedDelimiter{\floor}{\lfloor}{\rfloor}
\DeclarePairedDelimiter{\set}{\lbrace}{\rbrace}
\DeclarePairedDelimiter{\iprod}{\langle}{\rangle}

\DeclareMathOperator{\Int}{int}
\DeclareMathOperator{\mean}{mean}

% commonly used sets
\DeclareMathOperator{\N}{{\mathbb{N}}}
\DeclareMathOperator{\Q}{{\mathbb{Q}}}
\DeclareMathOperator{\Z}{{\mathbb{Z}}}
\DeclareMathOperator{\R}{{\mathbb{R}}}
\DeclareMathOperator{\C}{{\mathbb{C}}}
\DeclareMathOperator{\F}{{\mathbb{F}}}

\newcommand{\mbf}[1]{{\boldmath\bfseries #1}}

% proof implications
\newcommand{\imp}[2]{($#1\Rightarrow#2$)\hspace{0.2cm}}
\newcommand{\impe}[2]{($#1\Leftrightarrow#2$)\hspace{0.2cm}}
\newcommand{\impr}{{($\Rightarrow$)\hspace{0.2cm}}}
\newcommand{\impl}{{($\Leftarrow$)\hspace{0.2cm}}}

% align macros
\newcommand{\agspace}{\ensuremath{\phantom{--}}}
\newcommand{\agvdots}{\ensuremath{\hspace{0.16cm}\vdots}}

% convenient brackets
\newcommand{\brac}[1]{\ensuremath{\left\langle #1 \right\rangle}}

% arrows
\newcommand{\lto}[0]{\ensuremath{\longrightarrow}}
\newcommand{\fto}[1]{\ensuremath{\xrightarrow{\scriptstyle{#1}}}}
\newcommand{\hto}[0]{\ensuremath{\hookrightarrow}}
\newcommand{\mapsfrom}[0]{\mathrel{\reflectbox{\ensuremath{\mapsto}}}}

\DeclareMathOperator{\Ann}{Ann}
\DeclareMathOperator{\Aut}{Aut}
\DeclareMathOperator{\chr}{char}
\DeclareMathOperator{\coker}{coker}
\DeclareMathOperator{\disc}{disc}
\DeclareMathOperator{\End}{End}
\DeclareMathOperator{\Fix}{Fix}
\DeclareMathOperator{\Frac}{Frac}
\DeclareMathOperator{\Gal}{Gal}
\DeclareMathOperator{\GL}{GL}
\DeclareMathOperator{\Hom}{Hom}
\DeclareMathOperator{\id}{id}
\DeclareMathOperator{\im}{im}
\DeclareMathOperator{\lcm}{lcm}
\DeclareMathOperator{\Nil}{Nil}
\DeclareMathOperator{\Spec}{Spec}
\DeclareMathOperator{\spn}{span}
\DeclareMathOperator{\Stab}{Stab}
\DeclareMathOperator{\Tor}{Tor}

\newcommand{\sset}{\subseteq}

\theoremstyle{break}
\theorembodyfont{\upshape}

\newtheorem{thm}{Theorem}[subsection]
\tcolorboxenvironment{thm}{
enhanced jigsaw,
colframe=Dandelion,
colback=White!90!Dandelion,
drop fuzzy shadow east,
rightrule=2mm,
sharp corners,
before skip=10pt,after skip=10pt
}

\newtheorem{cor}{Corollary}[thm]
\tcolorboxenvironment{cor}{
boxrule=0pt,
boxsep=0pt,
colback={White!90!RoyalPurple},
enhanced jigsaw,
borderline west={2pt}{0pt}{RoyalPurple},
sharp corners,
before skip=10pt,
after skip=10pt,
breakable
}

\newtheorem{lem}[thm]{Lemma}
\tcolorboxenvironment{lem}{
enhanced jigsaw,
colframe=Red,
colback={White!95!Red},
rightrule=2mm,
sharp corners,
before skip=10pt,after skip=10pt
}

\newtheorem{ex}[thm]{Example}
\tcolorboxenvironment{ex}{% from ntheorem
blanker,left=5mm,
sharp corners,
before skip=10pt,after skip=10pt,
borderline west={2pt}{0pt}{Gray}
}

\newtheorem*{pf}{Proof}
\tcolorboxenvironment{pf}{% from ntheorem
breakable,blanker,left=5mm,
sharp corners,
before skip=10pt,after skip=10pt,
borderline west={2pt}{0pt}{NavyBlue!80!white}
}

\newtheorem{defn}{Definition}[subsection]
\tcolorboxenvironment{defn}{
enhanced jigsaw,
colframe=Cerulean,
colback=White!90!Cerulean,
drop fuzzy shadow east,
rightrule=2mm,
sharp corners,
before skip=10pt,after skip=10pt
}

\newtheorem{prop}[thm]{Proposition}
\tcolorboxenvironment{prop}{
boxrule=0pt,
boxsep=0pt,
colback={White!90!Green},
enhanced jigsaw,
borderline west={2pt}{0pt}{Green},
sharp corners,
before skip=10pt,
after skip=10pt,
breakable
}

\setlength\parindent{0pt}
\setlength{\parskip}{2pt}

\newcommand{\subject}{MATH 135 \\ Proof Techniques}
\newcommand{\semester}{Winter 2023}
\newcommand{\professor}{Jason Bell}

\begin{document}
\let\ref\Cref

\title{\subject}
\author{Sachin Kumar\thanks{\itshape skmuthuk@uwaterloo.ca}\\ University of Waterloo}
\date{\semester\thanks{Last updated: \today}}

\maketitle
\newpage
\tableofcontents
% \listoffigures
% \listoftables
\newpage






% main document ----------------------------------------------------------

% -------------------- CHAPTER 1 -------------------------
\section{Proving Mathematical Statements}
\subsection{Proving Universally Quantified Statements}
\subsubsection{Proof Technique 1}
To prove the universally quantified statement "$\forall x \in \S, P(x)$": \\

Choose a representative mathematical object $x \in \S$. This cannot be a specific object. 
It has to be a placeholder, that is, a variable, so that our argument would work for any 
specific member of the domain $S$.\\

Then, show that the open sentence $P$ must be true for our representative $x$, using known 
facts about the elements of $S$.

\subsubsection{Proof Technique 2}
To disprove the universally quantified statement "$\forall x \in \S, P(x)$": \\

Find an element $x \in S$ for which the open sentence $P(x)$ is false. This process is 
called finding a \textbf{counter-example}.

\subsection{Proving Existentially Quantified Statements}
\subsubsection{Proof Technique 3}
To prove the existentially quantified statement "$\exists x \in \S, P(x)$": \\

Provide an explicit value of $x$ from the domain $S$, and show that $P(x)$
is true for this value of $x$. In other words, find an element of $S$ that 
satisfies property $P$.

\subsubsection{Proof Technique 4}
To disprove the existentially quantified statement "$\exists x \in S, P(x)$": \\

Prove the universally quantified statement "$\forall x \in S, \neg P(x)$"

\subsection{Proving Implications}
\subsubsection{Proof Technique 5}
For proving an implication: 
\begin{enumerate}
    \item To prove the implication "$A \implies B$", \textbf{assume} that the hypothesis 
    $A$ is true, and use this assumption to show that the conclusion $B$ is true. The 
    hypothesis $A$ is what you start with. The conclusion $B$ is where you must end up.
    \item To prove the universally quantified implication "$\forall x \in S, P(x) \implies Q(x)$":\\
    Let $x$ be an arbitrary element of $S$, assume that the hypothesis $P(x)$ is true, and 
    use this assumption to show that the conclusion $Q(x)$ is true.
\end{enumerate}

\subsection{Proof by Contrapositive}
\subsubsection{Proof Technique 6}
For proving an implication using the contrapositive:
\begin{enumerate}
    \item To prove the implication "$A \implies B$", replace it with its contrapositive
    "$(\neg B) \implies (\neg A)$". Then prove this contrapositive, usually by a direct 
    proof. That is, assume $\neg B$ is true and deduce that $\neg A$ must be true as well.
    \item To prove the universally quantified implication 
    "$\forall x \in S, P(x) \implies Q(x)$", replace it with its universally quantified contrapositive
    "$\forall x \in S, (\neg Q(x)) \implies (\neg P(x))$". Then prove this universally quantified contrapositive.
\end{enumerate}
\end{document}
